% ============================================================
% CHAPITRE 7 : SPRINT 4 - PAIEMENT & INVENTAIRE
% ============================================================

\chapter{Sprint 4 : Paiement Stripe et Gestion d'Inventaire}

\section{Presentation du Sprint}

\subsection{Objectifs}

Le quatrieme sprint vise a integrer le systeme de paiement securise Stripe et a developper un module de gestion d'inventaire.

\begin{table}[htbp]
\centering
\caption{Fiche du Sprint 4}
\rowcolors{2}{lightgray}{white}
\begin{tabular}{|l|l|}
\hline
\rowcolor{darkblue}
\tableheader{Attribut} & \tableheader{Valeur} \\
\hline
Duree & 2 semaines \\
\hline
Theme & Paiement et Inventaire \\
\hline
Priorite & Haute \\
\hline
\end{tabular}
\end{table}

\section{Sprint Backlog}

\begin{table}[htbp]
\centering
\caption{Sprint Backlog - Sprint 4}
\rowcolors{2}{lightgray}{white}
\begin{tabular}{|c|p{7cm}|c|c|}
\hline
\rowcolor{darkblue}
\tableheader{ID} & \tableheader{User Story} & \tableheader{Priorite} & \tableheader{Points} \\
\hline
US-4.1 & En tant qu'utilisateur, je veux payer par carte bancaire & Haute & 8 \\
\hline
US-4.2 & En tant qu'utilisateur, je veux un paiement securise & Haute & 5 \\
\hline
US-4.3 & En tant qu'admin, je veux suivre les niveaux de stock & Haute & 5 \\
\hline
US-4.4 & En tant qu'admin, je veux des alertes de stock bas & Moyenne & 5 \\
\hline
US-4.5 & En tant qu'admin, je veux gerer les fournisseurs & Moyenne & 5 \\
\hline
US-4.6 & En tant qu'admin, je veux creer des commandes de reapprovisionnement & Moyenne & 5 \\
\hline
\end{tabular}
\end{table}

\section{Analyse et Conception}

\subsection{Diagramme de Cas d'Utilisation}

\begin{figure}[H]
\centering
\includegraphics[width=0.95\textwidth]{cas d'itlisation 4.png}
\caption{Diagramme de cas d'utilisation - Paiement et Inventaire}
\label{fig:sprint4_usecase}
\end{figure}

\textbf{Cas d'utilisation - Paiement :}
\begin{itemize}
    \item \textbf{\textcolor{darkblue}{Initier paiement :}} Demarrage du processus de paiement
    \item \textbf{\textcolor{darkblue}{Saisir carte :}} Entree des informations bancaires via Stripe
    \item \textbf{\textcolor{darkblue}{Confirmer paiement :}} Validation de la transaction
\end{itemize}

\textbf{Cas d'utilisation - Inventaire :}
\begin{itemize}
    \item \textbf{\textcolor{darkblue}{Consulter tableau de bord :}} Vue d'ensemble des stocks
    \item \textbf{\textcolor{darkblue}{Gerer mouvements :}} Enregistrement des entrees/sorties de stock
    \item \textbf{\textcolor{darkblue}{Gerer fournisseurs :}} CRUD sur les fournisseurs
    \item \textbf{\textcolor{darkblue}{Commander reapprovisionnement :}} Creation de commandes fournisseurs
\end{itemize}

\subsection{Diagramme de Classes}

\begin{figure}[H]
\centering
\includegraphics[width=0.95\textwidth]{classe 4.png}
\caption{Diagramme de classes - Payment, StockMovement, Supplier, PurchaseOrder et Product}
\label{fig:sprint4_class}
\end{figure}

\textbf{Description des entites :}
\begin{itemize}
    \item \textbf{\textcolor{navyblue}{Payment :}} Enregistrement d'un paiement. Attributs : id, amount, currency (TND), status (PENDING, COMPLETED, FAILED, REFUNDED), stripePaymentId, stripeSessionId, paymentMethod, timestamps. Relation OneToOne avec Order.
    \item \textbf{\textcolor{navyblue}{Order :}} Commande associee au paiement. Le paiement reference la commande pour laquelle il a ete effectue.
    \item \textbf{\textcolor{navyblue}{StockMovement :}} Historique des mouvements de stock. Attributs : id, quantity, type (IN, OUT, ADJUSTMENT), reason, reference, createdBy, createdAt. Relation ManyToOne avec Product.
    \item \textbf{\textcolor{navyblue}{Product :}} Produit dont le stock est affecte par les mouvements. Le stock est mis a jour automatiquement.
    \item \textbf{\textcolor{navyblue}{Supplier :}} Fournisseur de pieces. Attributs : id, name, email, phone, address, contactPerson, isActive.
    \item \textbf{\textcolor{navyblue}{PurchaseOrder :}} Commande de reapprovisionnement. Attributs : id, orderNumber, status, totalAmount, expectedDelivery, notes. Relation ManyToOne avec Supplier, OneToMany avec PurchaseOrderItem.
    \item \textbf{\textcolor{navyblue}{PurchaseOrderItem :}} Article d'une commande fournisseur. Attributs : id, quantity, unitPrice, receivedQuantity. Relation ManyToOne avec Product.
\end{itemize}

\subsection{Diagramme de Sequence}

\begin{figure}[H]
\centering
\includegraphics[width=0.95\textwidth]{sequence 4.png}
\caption{Diagramme de sequence - Paiement avec Stripe Checkout}
\label{fig:sprint4_sequence}
\end{figure}

\section{Realisation}

\subsection{Integration Stripe}

\begin{itemize}
    \item \textbf{\textcolor{steelblue}{Stripe Checkout :}} Interface hebergee par Stripe, conforme PCI-DSS
    \item \textbf{\textcolor{steelblue}{Webhooks :}} Endpoint securise pour confirmation asynchrone
    \item \textbf{\textcolor{steelblue}{Mode Test :}} Cles API de test pour simulation
\end{itemize}

\section{Bilan du Sprint}

\begin{table}[htbp]
\centering
\caption{Bilan Sprint 4}
\rowcolors{2}{lightgray}{white}
\begin{tabular}{|l|c|}
\hline
\rowcolor{darkblue}
\tableheader{Metrique} & \tableheader{Valeur} \\
\hline
User Stories terminees & 6/6 \\
\hline
Points realises & 33 \\
\hline
Velocite & 100\% \\
\hline
\end{tabular}
\end{table}
