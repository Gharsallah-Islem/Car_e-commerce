% ============================================================
% CHAPITRE 10 : TESTS ET VALIDATION
% ============================================================

\chapter{Tests et Validation}

\section{Introduction}

Ce chapitre presente la strategie de tests mise en place pour assurer la qualite et la fiabilite de notre plateforme. Nous avons adopte une approche multi-niveaux couvrant les tests unitaires, d'integration et de validation utilisateur.

\section{Strategie de Tests}

\subsection{Pyramide des Tests}

Notre strategie suit la pyramide des tests classique :

\diagrammePlaceholder{test_pyramid}{Pyramide des tests}

\begin{enumerate}
    \item \textbf{Tests Unitaires (Base) :} Testent les composants individuels en isolation
    \item \textbf{Tests d'Integration (Milieu) :} Verifient l'interaction entre les modules
    \item \textbf{Tests E2E (Sommet) :} Valident les parcours utilisateur complets
\end{enumerate}

\section{Tests Unitaires}

\subsection{Backend - JUnit et Mockito}

Les tests unitaires du backend couvrent :

\begin{itemize}
    \item \textbf{Services :} Logique metier (AuthService, ProductService, OrderService, etc.)
    \item \textbf{Utilitaires :} Fonctions de generation de tokens, validation, etc.
    \item \textbf{Entites :} Methodes de calcul et validations
\end{itemize}

\textbf{Outils utilises :}
\begin{itemize}
    \item JUnit 5 pour les assertions et le cycle de vie des tests
    \item Mockito pour le mocking des dependances
    \item Spring Boot Test pour le contexte d'application
\end{itemize}

\subsection{Frontend - Jasmine et Karma}

Les tests unitaires Angular couvrent :

\begin{itemize}
    \item \textbf{Services :} Appels HTTP, gestion d'etat
    \item \textbf{Composants :} Logique de presentation, interactions
    \item \textbf{Pipes et Directives :} Transformations de donnees
\end{itemize}

\subsection{Module IA - Pytest}

Les tests du module Python couvrent :

\begin{itemize}
    \item Verification du chargement du modele
    \item Tests de prediction avec images de reference
    \item Validation des endpoints API (health, predict)
\end{itemize}

\section{Tests d'Integration}

\subsection{Tests API}

Les tests d'integration verifient le bon fonctionnement des endpoints REST :

\begin{table}[htbp]
\centering
\caption{Resume des tests API}
\begin{tabular}{|l|c|c|}
\hline
\textbf{Module} & \textbf{Nombre de tests} & \textbf{Reussis} \\
\hline
Authentication & 8 & 8 \\
\hline
Products & 12 & 12 \\
\hline
Cart & 6 & 6 \\
\hline
Orders & 10 & 10 \\
\hline
Payments & 4 & 4 \\
\hline
Delivery & 8 & 8 \\
\hline
Chat & 6 & 6 \\
\hline
AI & 4 & 4 \\
\hline
\textbf{Total} & \textbf{58} & \textbf{58} \\
\hline
\end{tabular}
\end{table}

\subsection{Tests avec Postman}

Une collection Postman complete a ete creee pour tester manuellement et automatiquement toutes les API. La collection comprend :

\begin{itemize}
    \item Variables d'environnement (URLs, tokens)
    \item Pre-request scripts pour l'authentification automatique
    \item Tests automatises pour valider les reponses
    \item Collection Runner pour l'execution en lot
\end{itemize}

\section{Tests d'Acceptation Utilisateur}

\subsection{Scenarios de Test}

Les scenarios suivants ont ete valides avec des utilisateurs reels :

\begin{table}[htbp]
\centering
\caption{Scenarios de test utilisateur}
\begin{tabular}{|p{9cm}|c|}
\hline
\textbf{Scenario} & \textbf{Resultat} \\
\hline
Un utilisateur s'inscrit, verifie son email et se connecte & \checkmark \\
\hline
Un utilisateur recherche et filtre des produits par categorie & \checkmark \\
\hline
Un utilisateur ajoute des produits au panier et passe commande & \checkmark \\
\hline
Un utilisateur effectue un paiement via Stripe & \checkmark \\
\hline
Un utilisateur utilise l'IA pour identifier une piece & \checkmark \\
\hline
Un utilisateur suit sa livraison en temps reel & \checkmark \\
\hline
Un utilisateur contacte le support via chat & \checkmark \\
\hline
Un admin gere les produits et commandes & \checkmark \\
\hline
\end{tabular}
\end{table}

\section{Couverture de Code}

\begin{table}[htbp]
\centering
\caption{Couverture de code par module}
\begin{tabular}{|l|c|c|c|}
\hline
\textbf{Module} & \textbf{Lignes} & \textbf{Branches} & \textbf{Fonctions} \\
\hline
Backend Services & 78\% & 72\% & 85\% \\
\hline
Backend Controllers & 82\% & 75\% & 90\% \\
\hline
Frontend Services & 75\% & 68\% & 80\% \\
\hline
Module IA & 70\% & 65\% & 75\% \\
\hline
\end{tabular}
\end{table}

\section{Validation des Fonctionnalites}

\begin{table}[htbp]
\centering
\caption{Validation par plateforme}
\begin{tabular}{|l|c|c|c|}
\hline
\textbf{Module} & \textbf{Web} & \textbf{Mobile} & \textbf{Admin} \\
\hline
Authentification & \checkmark & \checkmark & \checkmark \\
\hline
Catalogue produits & \checkmark & \checkmark & \checkmark \\
\hline
Panier et commandes & \checkmark & \checkmark & \checkmark \\
\hline
Paiement Stripe & \checkmark & \checkmark & - \\
\hline
Module IA & \checkmark & \checkmark & - \\
\hline
Chat support & \checkmark & \checkmark & \checkmark \\
\hline
Suivi livraison & \checkmark & \checkmark & \checkmark \\
\hline
\end{tabular}
\end{table}

\section{Conclusion}

La strategie de tests mise en place nous a permis d'assurer un niveau de qualite eleve pour notre plateforme. Avec un taux de reussite de 100\% sur les tests automatises et une validation complete des scenarios utilisateur, nous avons confiance dans la stabilite et la fiabilite de l'application.
