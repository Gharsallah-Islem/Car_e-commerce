% ============================================================
% CHAPITRE 4 : SPRINT 1 - FONDATION & AUTHENTIFICATION
% ============================================================

\chapter{Sprint 1 : Fondation et Authentification}

\section{Presentation du Sprint}

\subsection{Objectifs}

Le premier sprint constitue la fondation de notre projet. Il vise a mettre en place l'infrastructure technique et le systeme d'authentification securise.

\begin{table}[htbp]
\centering
\caption{Fiche du Sprint 1}
\rowcolors{2}{lightgray}{white}
\begin{tabular}{|l|l|}
\hline
\rowcolor{darkblue}
\tableheader{Attribut} & \tableheader{Valeur} \\
\hline
Duree & 2 semaines \\
\hline
Theme & Fondation et Authentification \\
\hline
Priorite & Haute \\
\hline
\end{tabular}
\end{table}

\subsection{Objectifs Detailles}

\begin{enumerate}
    \item Configuration de l'environnement de developpement (Angular, Spring Boot, PostgreSQL)
    \item Mise en place de l'architecture du projet selon le pattern MVC
    \item Developpement du systeme d'authentification (inscription, connexion, deconnexion)
    \item Gestion des utilisateurs et des roles (CLIENT, ADMIN, DRIVER, SUPPORT, SUPER\_ADMIN)
    \item Securisation des API avec JWT (JSON Web Token)
    \item Verification d'email et reinitialisation de mot de passe
\end{enumerate}

\section{Sprint Backlog}

\begin{table}[htbp]
\centering
\caption{Sprint Backlog - Sprint 1}
\rowcolors{2}{lightgray}{white}
\begin{tabular}{|c|p{7cm}|c|c|}
\hline
\rowcolor{darkblue}
\tableheader{ID} & \tableheader{User Story} & \tableheader{Priorite} & \tableheader{Points} \\
\hline
US-1.1 & En tant qu'utilisateur, je veux creer un compte avec mon email et mot de passe & Haute & 5 \\
\hline
US-1.2 & En tant qu'utilisateur, je veux me connecter de maniere securisee & Haute & 5 \\
\hline
US-1.3 & En tant qu'utilisateur, je veux reinitialiser mon mot de passe oublie & Moyenne & 3 \\
\hline
US-1.4 & En tant qu'admin, je veux gerer les roles des utilisateurs & Haute & 5 \\
\hline
US-1.5 & En tant qu'utilisateur, je veux verifier mon email & Moyenne & 3 \\
\hline
US-1.6 & En tant qu'utilisateur, je veux mettre a jour mon profil & Moyenne & 3 \\
\hline
\end{tabular}
\end{table}

\section{Analyse et Conception}

\subsection{Diagramme de Cas d'Utilisation}

Le diagramme de cas d'utilisation suivant presente les fonctionnalites d'authentification et de gestion des utilisateurs. Il identifie les acteurs principaux : le Visiteur (non authentifie), l'Utilisateur authentifie, l'Administrateur avec privileges etendus, et le Systeme Email pour les notifications.

\begin{figure}[H]
\centering
\includegraphics[width=0.95\textwidth]{cas utilisation 1.png}
\caption{Diagramme de cas d'utilisation - Module Authentification}
\label{fig:sprint1_usecase}
\end{figure}

\textbf{Acteurs identifies :}
\begin{itemize}
    \item \textbf{\textcolor{darkblue}{Visiteur :}} Utilisateur non authentifie pouvant s'inscrire ou se connecter
    \item \textbf{\textcolor{darkblue}{Utilisateur :}} Utilisateur authentifie avec acces aux fonctionnalites de base
    \item \textbf{\textcolor{darkblue}{Administrateur :}} Utilisateur avec privileges de gestion complete
    \item \textbf{\textcolor{darkblue}{Systeme :}} Responsable de l'envoi d'emails et de la validation des tokens
\end{itemize}

\subsection{Diagramme de Classes}

Le diagramme de classes ci-dessous presente l'architecture des entites liees a la gestion des utilisateurs et a l'authentification.

\begin{figure}[H]
\centering
\includegraphics[width=0.95\textwidth]{classe 1.png}
\caption{Diagramme de classes - Entites User et Role}
\label{fig:sprint1_class}
\end{figure}

\textbf{Description des entites :}
\begin{itemize}
    \item \textbf{\textcolor{navyblue}{User :}} Entite principale representant un utilisateur du systeme. Elle contient les attributs : id (UUID), email, password (hash BCrypt), fullName, phone, address, profilePicture, isActive, isEmailVerified, emailVerificationToken, passwordResetToken, et les timestamps de creation/modification.
    \item \textbf{\textcolor{navyblue}{Role :}} Entite definissant les differents roles possibles dans le systeme. Les roles disponibles sont : CLIENT (client standard), ADMIN (administrateur), DRIVER (livreur), SUPPORT (agent de support), et SUPER\_ADMIN (super administrateur avec tous les privileges). Relation ManyToOne avec User.
\end{itemize}

\subsection{Diagramme de Sequence}

Le diagramme de sequence suivant illustre le processus complet de connexion avec generation et validation du token JWT.

\begin{figure}[H]
\centering
\includegraphics[width=0.95\textwidth]{sequence 1.png}
\caption{Diagramme de sequence - Processus de connexion JWT}
\label{fig:sprint1_sequence}
\end{figure}

\textbf{Description du flux de connexion :}
\begin{enumerate}
    \item L'utilisateur saisit ses identifiants (email, mot de passe) dans le formulaire
    \item Le frontend Angular envoie une requete POST vers /api/auth/login
    \item Le backend verifie les identifiants avec BCrypt
    \item Si valides, un token JWT est genere avec les claims (sub, role, exp)
    \item Le token est stocke dans localStorage cote client
    \item Les requetes suivantes incluent le token dans l'en-tete Authorization
\end{enumerate}

\section{Realisation}

\subsection{Architecture d'Authentification}

L'architecture d'authentification mise en place repose sur plusieurs composants :
\begin{itemize}
    \item \textbf{\textcolor{steelblue}{SecurityConfig :}} Configuration Spring Security avec routes publiques et protegees
    \item \textbf{\textcolor{steelblue}{JwtAuthenticationFilter :}} Filtre interceptant chaque requete pour valider le JWT
    \item \textbf{\textcolor{steelblue}{AuthService :}} Logique metier d'inscription, connexion et reinitialisation
\end{itemize}

\subsection{Gestion des Roles}

\begin{table}[htbp]
\centering
\caption{Roles et permissions}
\rowcolors{2}{lightgray}{white}
\begin{tabular}{|l|p{7cm}|}
\hline
\rowcolor{darkblue}
\tableheader{Role} & \tableheader{Permissions} \\
\hline
CLIENT & Acces au catalogue, panier, commandes, profil \\
\hline
ADMIN & Gestion produits, commandes, utilisateurs, analytics \\
\hline
DRIVER & Acces aux livraisons assignees, mise a jour des statuts \\
\hline
SUPPORT & Acces au chat support, gestion des reclamations \\
\hline
SUPER\_ADMIN & Tous les privileges, gestion des admins, configuration systeme \\
\hline
\end{tabular}
\end{table}

\section{Bilan du Sprint}

\begin{table}[htbp]
\centering
\caption{Bilan Sprint 1}
\rowcolors{2}{lightgray}{white}
\begin{tabular}{|l|c|}
\hline
\rowcolor{darkblue}
\tableheader{Metrique} & \tableheader{Valeur} \\
\hline
User Stories planifiees & 6 \\
\hline
User Stories terminees & 6 \\
\hline
Points planifies & 24 \\
\hline
Points realises & 24 \\
\hline
Velocite & 100\% \\
\hline
\end{tabular}
\end{table}

Le Sprint 1 a ete complete avec succes, etablissant une base solide pour les sprints suivants.
