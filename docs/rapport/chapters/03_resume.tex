% ============================================================
% RESUME (sans Abstract anglais)
% ============================================================

\chapter*{Resume}
\addcontentsline{toc}{chapter}{Resume}

\vspace{0.5cm}

Ce projet de fin d'etudes porte sur la conception et la realisation d'une plateforme e-commerce intelligente dediee a la vente de pieces detachees automobiles. L'objectif principal est de faciliter l'identification et l'achat de pieces auto grace a l'integration de technologies d'Intelligence Artificielle.

\vspace{0.5cm}

La plateforme developpee offre une solution complete comprenant :
\begin{itemize}
    \item Une application web responsive (Angular 18)
    \item Une application mobile native (Kotlin Android)
    \item Un panneau d'administration pour la gestion des produits et des commandes
    \item Un systeme de paiement securise via Stripe
    \item Un module de reconnaissance d'images base sur les reseaux de neurones convolutifs (CNN)
    \item Un systeme de recommandation personnalise
    \item Un chat de support en temps reel
    \item Un systeme de suivi de livraison avec cartographie
\end{itemize}

\vspace{0.5cm}

Le developpement a ete realise en suivant la methodologie Scrum, organise en six sprints couvrant l'authentification, la gestion des produits, les commandes, les paiements, l'intelligence artificielle et le systeme de livraison.

\vspace{0.5cm}

Les technologies utilisees incluent Angular 18 pour le frontend web, Kotlin pour l'application mobile Android, Spring Boot 3 pour le backend, PostgreSQL pour la base de donnees, FastAPI et TensorFlow pour le module d'IA, et Leaflet avec OpenRouteService pour la cartographie.

\vspace{1cm}

\textbf{Mots-cles :} E-commerce, Pieces automobiles, Intelligence Artificielle, Reconnaissance d'images, Spring Boot, Angular, Kotlin, Systeme de recommandation, Livraison en temps reel.

\newpage
