% ============================================================
% Rapport de Projet d'Integration
% Plateforme E-Commerce Intelligente pour Pieces Automobiles
% ============================================================

\documentclass[12pt,a4paper,french]{report}

% ============================================================
% PACKAGES
% ============================================================

% Encodage et langue
\usepackage[utf8]{inputenc}
\usepackage[T1]{fontenc}
\usepackage[french]{babel}

% Mise en page
\usepackage[top=2.5cm, bottom=2.5cm, left=2.5cm, right=2.5cm]{geometry}
\usepackage{setspace}
\onehalfspacing

% Polices
\usepackage{lmodern}

% Graphiques et couleurs
\usepackage{graphicx}
\graphicspath{{images/}}
\usepackage{xcolor}
\usepackage{float}

% ============================================================
% DEFINITION DES COULEURS PROFESSIONNELLES
% ============================================================

% Couleurs principales
\definecolor{darkblue}{RGB}{0, 51, 102}
\definecolor{navyblue}{RGB}{0, 63, 114}
\definecolor{steelblue}{RGB}{70, 130, 180}
\definecolor{darkgray}{RGB}{64, 64, 64}
\definecolor{mediumgray}{RGB}{128, 128, 128}
\definecolor{lightgray}{RGB}{245, 245, 245}
\definecolor{accentblue}{RGB}{41, 128, 185}

% ============================================================
% STYLE DES TITRES (titlesec)
% ============================================================

\usepackage{titlesec}

% Style des chapitres
\titleformat{\chapter}[display]
{\normalfont\Huge\bfseries\color{darkblue}}
{\chaptertitlename\ \thechapter}{20pt}{\Huge\color{darkblue}}
\titlespacing*{\chapter}{0pt}{-20pt}{40pt}

% Style des sections
\titleformat{\section}
{\normalfont\Large\bfseries\color{navyblue}}
{\thesection}{1em}{}
\titlespacing*{\section}{0pt}{3.5ex plus 1ex minus .2ex}{2.3ex plus .2ex}

% Style des sous-sections
\titleformat{\subsection}
{\normalfont\large\bfseries\color{steelblue}}
{\thesubsection}{1em}{}

% Style des sous-sous-sections
\titleformat{\subsubsection}
{\normalfont\normalsize\bfseries\color{darkgray}}
{\thesubsubsection}{1em}{}

% ============================================================
% TABLEAUX AVEC COULEURS
% ============================================================

\usepackage{array}
\usepackage{tabularx}
\usepackage{booktabs}
\usepackage{multirow}
\usepackage{longtable}
\usepackage{colortbl}

% Style pour les en-tetes de tableaux
\newcommand{\tableheader}[1]{\textbf{\textcolor{white}{#1}}}
\newcommand{\headerrow}{\rowcolor{darkblue}}

% ============================================================
% LISTES
% ============================================================

\usepackage{enumitem}
\setlist[itemize]{leftmargin=*}
\setlist[enumerate]{leftmargin=*}

% ============================================================
% SYMBOLES
% ============================================================

\usepackage{amssymb}
\usepackage{pifont}

% ============================================================
% HYPERLIENS
% ============================================================

\usepackage{hyperref}
\hypersetup{
    colorlinks=true,
    linkcolor=darkblue,
    filecolor=magenta,
    urlcolor=accentblue,
    citecolor=darkblue,
    pdfborder={0 0 0}
}

% ============================================================
% EN-TETES ET PIEDS DE PAGE
% ============================================================

\usepackage{fancyhdr}
\pagestyle{fancy}
\fancyhf{}
\fancyhead[L]{\textcolor{darkgray}{\leftmark}}
\fancyhead[R]{\textcolor{darkblue}{\thepage}}
\renewcommand{\headrulewidth}{0.5pt}
\renewcommand{\headrule}{\hbox to\headwidth{\color{darkblue}\leaders\hrule height \headrulewidth\hfill}}

% Style pour les pages de debut de chapitre
\fancypagestyle{plain}{
    \fancyhf{}
    \fancyfoot[C]{\textcolor{darkblue}{\thepage}}
    \renewcommand{\headrulewidth}{0pt}
}

% ============================================================
% BIBLIOGRAPHIE
% ============================================================

\usepackage{cite}

% ============================================================
% BOITES COLOREES POUR MISE EN VALEUR
% ============================================================

\usepackage{tcolorbox}
\tcbuselibrary{skins,breakable}

\newtcolorbox{infobox}{
    colback=lightgray,
    colframe=darkblue,
    fonttitle=\bfseries,
    title=Information,
    arc=3mm,
    boxrule=1pt
}

% ============================================================
% DEBUT DU DOCUMENT
% ============================================================

\begin{document}

% Page de garde
% ============================================================
% PAGE DE GARDE
% ============================================================

\begin{titlepage}
    \begin{center}
        
        % En-tete officiel
        {\large \textbf{\textcolor{darkblue}{Republique Tunisienne}}}\\[0.2cm]
        {\normalsize \textcolor{darkgray}{Ministere de l'Enseignement Superieur et de la Recherche Scientifique}}\\[0.3cm]
        {\normalsize \textbf{\textcolor{darkblue}{Direction Generale des Etudes Technologiques}}}\\[0.8cm]
        
        % Logo ISET
        \includegraphics[width=3cm]{iset logo.png}
        \vspace{0.5cm}
        
        % Nom de l'institution
        {\Large \textbf{\textcolor{darkblue}{Institut Superieur des Etudes Technologiques de Rades}}}\\[0.3cm]
        {\large \textcolor{navyblue}{Departement : Technologies de l'Informatique}}\\[1.5cm]
        
        % Ligne de separation
        {\color{darkblue}\rule{\textwidth}{2pt}}\\[0.5cm]
        
        % Titre du rapport
        {\LARGE \textbf{\textcolor{darkblue}{Rapport de Projet d'Integration}}}\\[0.8cm]
        
        {\Large \textbf{\textcolor{navyblue}{Conception et Realisation d'une\\[0.3cm]
        Plateforme E-Commerce Intelligente\\[0.3cm]
        pour Pieces Detachees Automobiles}}}\\[0.5cm]
        
        {\large \textit{\textcolor{steelblue}{avec Intelligence Artificielle et Systeme de Livraison Integre}}}\\[0.5cm]
        
        % Ligne de separation
        {\color{darkblue}\rule{\textwidth}{2pt}}\\[1.5cm]
        
        % Realise par
        \begin{minipage}{0.45\textwidth}
            \begin{flushleft}
                {\large \textbf{\textcolor{darkblue}{Realise par :}}}\\[0.3cm]
                \textcolor{darkgray}{Gharsallah Islem}\\
                \textcolor{darkgray}{Ben Jemaa Mohamed Malek}\\
                \textcolor{darkgray}{Hammi Youssef}
            \end{flushleft}
        \end{minipage}
        \hfill
        \begin{minipage}{0.45\textwidth}
            \begin{flushright}
                {\large \textbf{\textcolor{darkblue}{Encadre par :}}}\\[0.3cm]
                \textcolor{darkgray}{Mme. Lamia Mansouri}
            \end{flushright}
        \end{minipage}
        
        \vfill
        
        % Annee universitaire
        {\large \textbf{\textcolor{darkblue}{Annee Universitaire : 2024 - 2025}}}
        
    \end{center}
\end{titlepage}


% Pages preliminaires (numerotation romaine)
\pagenumbering{roman}
\setcounter{page}{1}

% ============================================================
% DÉDICACE
% ============================================================

\chapter*{Dédicace}
\addcontentsline{toc}{chapter}{Dédicace}

\vspace{2cm}

\begin{center}
\textit{\large À nos chers parents,}\\[0.5cm]
\textit{\large qui nous ont soutenus tout au long de notre parcours,}\\[0.5cm]
\textit{\large et qui ont fait de nous ce que nous sommes aujourd'hui.}\\[1cm]

\textit{\large À nos familles,}\\[0.5cm]
\textit{\large pour leur amour inconditionnel et leur encouragement constant.}\\[1cm]

\textit{\large À nos amis,}\\[0.5cm]
\textit{\large pour les moments partagés et le soutien moral.}\\[1cm]

\textit{\large À tous ceux qui ont contribué, de près ou de loin,}\\[0.5cm]
\textit{\large à la réalisation de ce projet.}\\[1.5cm]

\textbf{Nous vous dédions ce travail.}
\end{center}

\newpage

% ============================================================
% REMERCIEMENTS
% ============================================================

\chapter*{Remerciements}
\addcontentsline{toc}{chapter}{Remerciements}

\vspace{1cm}

Au terme de ce projet de fin d'études, nous tenons à exprimer notre profonde gratitude à toutes les personnes qui ont contribué à sa réalisation.

\vspace{0.5cm}

Nous adressons nos sincères remerciements à notre encadrante, \textbf{Mme. Lamia Mansouri}, pour sa disponibilité, ses conseils précieux et son accompagnement tout au long de ce projet. Son expertise et ses orientations ont été déterminantes pour la réussite de ce travail.

\vspace{0.5cm}

Nous remercions également l'ensemble du corps enseignant du \textbf{Département Technologies de l'Informatique} de l'\textbf{ISET de Radès} pour la formation de qualité qu'ils nous ont dispensée durant ces années d'études.

\vspace{0.5cm}

Nos remerciements vont aussi aux membres du jury qui ont accepté d'évaluer ce travail et de nous faire bénéficier de leurs remarques constructives.

\vspace{0.5cm}

Enfin, nous exprimons notre reconnaissance à nos familles et amis pour leur soutien moral et leurs encouragements constants qui nous ont permis de mener à bien ce projet.

\vspace{1.5cm}

\begin{flushright}
\textit{Gharsallah Islem}\\
\textit{Ben Jemaa Mohamed Malek}\\
\textit{Hammi Youssef}
\end{flushright}

\newpage

% ============================================================
% RESUME (sans Abstract anglais)
% ============================================================

\chapter*{Resume}
\addcontentsline{toc}{chapter}{Resume}

\vspace{0.5cm}

Ce projet de fin d'etudes porte sur la conception et la realisation d'une plateforme e-commerce intelligente dediee a la vente de pieces detachees automobiles. L'objectif principal est de faciliter l'identification et l'achat de pieces auto grace a l'integration de technologies d'Intelligence Artificielle.

\vspace{0.5cm}

La plateforme developpee offre une solution complete comprenant :
\begin{itemize}
    \item Une application web responsive (Angular 18)
    \item Une application mobile native (Kotlin Android)
    \item Un panneau d'administration pour la gestion des produits et des commandes
    \item Un systeme de paiement securise via Stripe
    \item Un module de reconnaissance d'images base sur les reseaux de neurones convolutifs (CNN)
    \item Un systeme de recommandation personnalise
    \item Un chat de support en temps reel
    \item Un systeme de suivi de livraison avec cartographie
\end{itemize}

\vspace{0.5cm}

Le developpement a ete realise en suivant la methodologie Scrum, organise en six sprints couvrant l'authentification, la gestion des produits, les commandes, les paiements, l'intelligence artificielle et le systeme de livraison.

\vspace{0.5cm}

Les technologies utilisees incluent Angular 18 pour le frontend web, Kotlin pour l'application mobile Android, Spring Boot 3 pour le backend, PostgreSQL pour la base de donnees, FastAPI et TensorFlow pour le module d'IA, et Leaflet avec OpenRouteService pour la cartographie.

\vspace{1cm}

\textbf{Mots-cles :} E-commerce, Pieces automobiles, Intelligence Artificielle, Reconnaissance d'images, Spring Boot, Angular, Kotlin, Systeme de recommandation, Livraison en temps reel.

\newpage


% Table des matieres
\tableofcontents

% Liste des figures
\listoffigures

% Liste des tableaux
\listoftables

% Contenu principal (numerotation arabe)
\pagenumbering{arabic}
\setcounter{page}{1}

% Introduction Generale
% ============================================================
% CHAPITRE 1 : INTRODUCTION GÉNÉRALE
% ============================================================

\chapter{Introduction Générale}

\section{Contexte du Projet}

Le secteur de l'e-commerce connaît une croissance exponentielle à l'échelle mondiale, transformant profondément les habitudes de consommation. En Tunisie, comme dans de nombreux pays, cette transformation numérique s'accélère, offrant de nouvelles opportunités aux entreprises et aux consommateurs.

Le marché des pièces détachées automobiles représente un segment particulièrement dynamique de l'économie. Avec un parc automobile en constante évolution et une demande croissante de pièces de rechange, ce secteur fait face à des défis majeurs : la diversité des modèles de véhicules, la complexité de l'identification des pièces compatibles, et la nécessité de garantir la qualité des produits.

Traditionnellement, les automobilistes doivent se rendre chez des garagistes ou des revendeurs spécialisés pour identifier et acheter les pièces dont ils ont besoin. Ce processus est souvent long, fastidieux et peut conduire à des erreurs coûteuses lorsque la pièce commandée n'est pas compatible avec le véhicule.

L'émergence de l'Intelligence Artificielle (IA) et des technologies de reconnaissance d'images ouvre de nouvelles perspectives pour résoudre ces problématiques. En permettant aux utilisateurs d'identifier des pièces automobiles simplement en prenant une photo, ces technologies peuvent révolutionner l'expérience d'achat dans ce domaine.

\section{Problématique}

Les acheteurs de pièces détachées automobiles sont confrontés à plusieurs difficultés majeures :

\begin{itemize}
    \item \textbf{Difficulté d'identification :} La plupart des automobilistes ne connaissent pas le nom exact des pièces dont ils ont besoin, ce qui complique leur recherche.
    
    \item \textbf{Problème de compatibilité :} Chaque véhicule a des spécifications propres, et une pièce qui semble identique peut ne pas être compatible avec tous les modèles.
    
    \item \textbf{Fragmentation du marché :} Les pièces sont dispersées entre de nombreux fournisseurs, rendant la comparaison des prix et de la disponibilité difficile.
    
    \item \textbf{Manque de conseil :} Les plateformes e-commerce classiques ne fournissent pas de recommandations personnalisées basées sur les symptômes ou les besoins spécifiques de l'utilisateur.
    
    \item \textbf{Suivi de livraison limité :} Le suivi des commandes est souvent rudimentaire, sans visibilité en temps réel sur la position du livreur.
\end{itemize}

Face à ces constats, la question qui se pose est la suivante : \textit{Comment concevoir une plateforme e-commerce intelligente capable d'assister les utilisateurs dans l'identification, le choix et l'achat de pièces détachées automobiles, tout en offrant une expérience utilisateur optimale ?}

\section{Objectifs du Projet}

Notre projet vise à développer une plateforme e-commerce complète et intelligente pour les pièces détachées automobiles. Les objectifs principaux sont :

\subsection{Objectifs Fonctionnels}

\begin{enumerate}
    \item \textbf{Système d'authentification sécurisé :} Permettre aux utilisateurs de créer un compte, se connecter et gérer leur profil de manière sécurisée.
    
    \item \textbf{Catalogue de produits complet :} Offrir un catalogue riche de pièces automobiles avec des fonctionnalités de recherche avancée et de filtrage.
    
    \item \textbf{Gestion du panier et des commandes :} Permettre aux utilisateurs d'ajouter des produits au panier, de passer des commandes et de suivre leur historique.
    
    \item \textbf{Paiement sécurisé :} Intégrer un système de paiement en ligne sécurisé via Stripe.
    
    \item \textbf{Reconnaissance d'images par IA :} Développer un module de reconnaissance d'images capable d'identifier les pièces automobiles à partir de photos.
    
    \item \textbf{Système de recommandation :} Proposer des recommandations personnalisées basées sur le comportement de l'utilisateur et l'analyse des symptômes.
    
    \item \textbf{Chat de support :} Mettre en place un système de chat en temps réel pour l'assistance client.
    
    \item \textbf{Suivi de livraison :} Offrir un suivi en temps réel des livraisons avec visualisation sur carte.
\end{enumerate}

\subsection{Objectifs Techniques}

\begin{enumerate}
    \item Adopter une architecture moderne et scalable basée sur les microservices.
    
    \item Utiliser les meilleures pratiques de développement (Clean Code, Design Patterns).
    
    \item Assurer la sécurité de l'application (authentification JWT, HTTPS, validation des données).
    
    \item Développer une interface utilisateur responsive et accessible (Web et Mobile).
    
    \item Intégrer des services tiers (Stripe, OpenRouteService) de manière robuste.
\end{enumerate}

\section{Périmètre et Limites}

\subsection{Périmètre du Projet}

Le projet couvre les aspects suivants :

\begin{itemize}
    \item Application web complète avec panneau d'administration
    \item Application mobile native Android (Kotlin) pour les utilisateurs
    \item Backend API RESTful
    \item Module d'Intelligence Artificielle pour la reconnaissance d'images
    \item Système de livraison simulé avec suivi en temps réel
\end{itemize}

\subsection{Limites}

Certaines fonctionnalités ne sont pas incluses dans le périmètre initial :

\begin{itemize}
    \item Les paiements réels sont simulés via le mode test de Stripe
    \item Le système de livraison utilise une simulation de mouvement du livreur
    \item L'application mobile n'inclut pas le panneau d'administration
    \item Le modèle d'IA est entraîné sur un jeu de données limité (50 classes de pièces)
\end{itemize}

\section{Organisation du Rapport}

Ce rapport est organisé en plusieurs chapitres :

\begin{itemize}
    \item \textbf{Chapitre 2 - État de l'Art :} Présente une étude des solutions existantes et des technologies utilisées.
    
    \item \textbf{Chapitre 3 - Méthodologie :} Décrit la méthodologie Scrum adoptée et l'environnement de développement.
    
    \item \textbf{Chapitres 4 à 9 - Sprints :} Détaillent le développement de chaque sprint avec les user stories, les fonctionnalités implémentées et les diagrammes UML.
    
    \item \textbf{Chapitre 10 - Tests et Validation :} Présente la stratégie de tests et les résultats de validation.
    
    \item \textbf{Chapitre 11 - Conclusion :} Synthétise les réalisations et propose des perspectives d'amélioration.
\end{itemize}


% Etat de l'Art
% ============================================================
% CHAPITRE 2 : ÉTAT DE L'ART
% ============================================================

\chapter{État de l'Art}

\section{Introduction}

Ce chapitre présente une étude approfondie des solutions existantes dans le domaine de l'e-commerce automobile, ainsi qu'une analyse des technologies qui seront utilisées dans notre projet. Cette étude nous permettra de positionner notre solution et de justifier nos choix technologiques.

\section{Étude des Plateformes E-Commerce Existantes}

\subsection{Plateformes Internationales}

\subsubsection{Amazon Auto Parts}

Amazon propose une section dédiée aux pièces automobiles avec plusieurs fonctionnalités intéressantes :

\begin{itemize}
    \item Recherche par compatibilité véhicule (marque, modèle, année)
    \item Large catalogue de produits avec avis clients
    \item Système de recommandation basé sur l'historique d'achat
    \item Livraison rapide via Amazon Prime
\end{itemize}

\textbf{Limites :} Pas de reconnaissance d'images, interface générique non spécialisée.

\subsubsection{RockAuto}

RockAuto est une référence dans le domaine des pièces automobiles en ligne :

\begin{itemize}
    \item Spécialisation dans les pièces automobiles
    \item Catalogue très complet avec diagrammes techniques
    \item Prix compétitifs avec multiple fournisseurs
    \item Recherche avancée par numéro de pièce OEM
\end{itemize}

\textbf{Limites :} Interface datée, pas d'IA, pas d'application mobile native.

\subsubsection{eBay Motors}

eBay Motors offre une marketplace pour les pièces automobiles :

\begin{itemize}
    \item Pièces neuves et d'occasion
    \item Système d'enchères et d'achat immédiat
    \item Garantie acheteur
    \item Vendeurs particuliers et professionnels
\end{itemize}

\textbf{Limites :} Qualité variable des vendeurs, pas de reconnaissance d'images.

\subsection{Plateformes Régionales et Tunisiennes}

En Tunisie, le marché des pièces automobiles en ligne est encore émergent. Les solutions existantes se limitent généralement à :

\begin{itemize}
    \item Des pages Facebook de revendeurs
    \item Des sites vitrine sans fonctionnalités e-commerce complètes
    \item Des applications basiques sans intelligence artificielle
\end{itemize}

\subsection{Tableau Comparatif}

\begin{table}[htbp]
\centering
\caption{Comparaison des plateformes existantes}
\label{tab:comparaison_plateformes}
\begin{tabular}{|l|c|c|c|c|c|}
\hline
\textbf{Fonctionnalité} & \textbf{Amazon} & \textbf{RockAuto} & \textbf{eBay} & \textbf{Local} & \textbf{Notre Solution} \\
\hline
Catalogue complet & \checkmark & \checkmark & \checkmark & -- & \checkmark \\
\hline
Application mobile & \checkmark & -- & \checkmark & -- & \checkmark \\
\hline
Reconnaissance IA & -- & -- & -- & -- & \checkmark \\
\hline
Recommandations IA & \checkmark & -- & -- & -- & \checkmark \\
\hline
Chat support & \checkmark & -- & \checkmark & -- & \checkmark \\
\hline
Suivi temps réel & \checkmark & -- & -- & -- & \checkmark \\
\hline
\end{tabular}
\end{table}

\section{Intelligence Artificielle dans l'E-Commerce}

\subsection{Reconnaissance d'Images}

La reconnaissance d'images est devenue un outil puissant dans l'e-commerce moderne. Elle permet :

\begin{itemize}
    \item \textbf{Recherche visuelle :} L'utilisateur peut prendre une photo d'un produit pour le rechercher.
    \item \textbf{Identification de pièces :} Particulièrement utile dans l'automobile où les pièces sont difficiles à nommer.
    \item \textbf{Contrôle qualité :} Détection automatique des défauts sur les produits.
\end{itemize}

\subsubsection{Réseaux de Neurones Convolutifs (CNN)}

Les CNN sont la technologie de référence pour la classification d'images. Les architectures les plus utilisées incluent :

\begin{itemize}
    \item \textbf{VGG16/VGG19 :} Architecture simple mais efficace, souvent utilisée comme baseline.
    \item \textbf{ResNet :} Introduit les connexions résiduelles pour les réseaux profonds.
    \item \textbf{EfficientNet :} Optimise le compromis entre précision et efficacité computationnelle.
    \item \textbf{MobileNet :} Conçu pour les appareils mobiles avec ressources limitées.
\end{itemize}

Pour notre projet, nous avons choisi \textbf{EfficientNetB0} car il offre un excellent équilibre entre précision et taille du modèle, ce qui est crucial pour un déploiement en production.

\subsection{Systèmes de Recommandation}

Les systèmes de recommandation sont essentiels pour personnaliser l'expérience utilisateur. On distingue trois approches principales :

\begin{enumerate}
    \item \textbf{Filtrage collaboratif :} Recommande des produits basés sur les préférences d'utilisateurs similaires.
    
    \item \textbf{Filtrage basé sur le contenu :} Recommande des produits similaires à ceux déjà consultés ou achetés.
    
    \item \textbf{Approche hybride :} Combine les deux méthodes pour de meilleures recommandations.
\end{enumerate}

Notre système utilisera une approche hybride, combinant l'analyse du comportement utilisateur avec les caractéristiques des produits.

\section{Technologies et Frameworks}

\subsection{Frontend : Angular}

Angular est un framework développé par Google pour la création d'applications web SPA (Single Page Application).

\textbf{Avantages :}
\begin{itemize}
    \item Architecture modulaire et maintenable
    \item TypeScript pour un typage fort
    \item Système de composants réutilisables
    \item Injection de dépendances intégrée
    \item Écosystème riche (Angular Material, RxJS)
\end{itemize}

\subsection{Mobile : Kotlin Android}

Kotlin est le langage officiel recommandé par Google pour le développement d'applications Android natives.

\textbf{Avantages :}
\begin{itemize}
    \item Langage moderne et concis
    \item Interopérabilité complète avec Java
    \item Null safety intégré
    \item Coroutines pour la programmation asynchrone
    \item Support officiel de Google et JetBrains
\end{itemize}

Notre application mobile Android utilise Kotlin avec les composants :
\begin{itemize}
    \item Retrofit pour les appels API REST vers le backend Spring Boot
    \item Coroutines pour la gestion asynchrone
    \item View Binding pour la liaison des vues
    \item Material Design 3 pour l'interface utilisateur
\end{itemize}

\subsection{Backend : Spring Boot}

Spring Boot est un framework Java qui simplifie le développement d'applications d'entreprise.

\textbf{Avantages :}
\begin{itemize}
    \item Configuration automatique
    \item Sécurité intégrée (Spring Security)
    \item Support JPA/Hibernate pour l'ORM
    \item Écosystème mature et documenté
    \item WebSocket pour la communication temps réel
\end{itemize}

\subsection{Base de Données : PostgreSQL}

PostgreSQL est un système de gestion de base de données relationnelle open source.

\textbf{Avantages :}
\begin{itemize}
    \item Fiabilité et robustesse
    \item Support des types de données complexes (JSON, arrays)
    \item Performances optimisées pour les grandes bases
    \item Conformité ACID
\end{itemize}

\subsection{Module IA : FastAPI et TensorFlow}

\textbf{FastAPI :}
\begin{itemize}
    \item Framework Python moderne et rapide
    \item Documentation automatique (OpenAPI/Swagger)
    \item Support asynchrone natif
    \item Validation automatique des données
\end{itemize}

\textbf{TensorFlow/Keras :}
\begin{itemize}
    \item Bibliothèque de deep learning de Google
    \item Support GPU pour l'entraînement
    \item Large communauté et documentation
    \item Modèles pré-entraînés disponibles
\end{itemize}

\subsection{Autres Technologies}

\begin{table}[htbp]
\centering
\caption{Technologies complémentaires utilisées}
\label{tab:technologies}
\begin{tabular}{|l|l|l|}
\hline
\textbf{Composant} & \textbf{Technologie} & \textbf{Utilisation} \\
\hline
Paiement & Stripe & Transactions sécurisées \\
\hline
Cartographie & Leaflet + OpenRouteService & Suivi de livraison \\
\hline
Temps réel & WebSocket (STOMP) & Chat, notifications \\
\hline
Authentification & JWT & Tokens sécurisés \\
\hline
Conteneurisation & Docker & Déploiement \\
\hline
Versioning & Git/GitHub & Gestion du code source \\
\hline
\end{tabular}
\end{table}

\section{Conclusion}

Cette étude de l'état de l'art nous a permis d'identifier les lacunes des solutions existantes et de définir les caractéristiques différenciatrices de notre plateforme. Notre solution se distingue par l'intégration de l'intelligence artificielle pour la reconnaissance d'images et les recommandations, ainsi que par un système de livraison avec suivi en temps réel.


% Methodologie
% ============================================================
% CHAPITRE 3 : MÉTHODOLOGIE ET GESTION DE PROJET
% ============================================================

\chapter{Méthodologie et Gestion de Projet}

\section{Introduction}

Ce chapitre présente la méthodologie de développement adoptée pour notre projet, ainsi que les outils et l'environnement de travail utilisés. L'organisation rigoureuse du projet est essentielle pour garantir la qualité des livrables et le respect des délais.

\section{Méthodologie Scrum}

\subsection{Présentation de Scrum}

Scrum est un cadre de travail agile qui permet de gérer des projets complexes de manière itérative et incrémentale. Il est particulièrement adapté aux projets de développement logiciel où les exigences peuvent évoluer.

\imagePlaceholder{scrum_framework}{Cadre de travail Scrum}

\subsection{Rôles Scrum}

Notre équipe est organisée selon les rôles Scrum suivants :

\begin{itemize}
    \item \textbf{Product Owner :} Mme. Lamia Mansouri (encadrante) - Définit les priorités et valide les fonctionnalités.
    
    \item \textbf{Scrum Master :} Gharsallah Islem - Facilite les cérémonies Scrum et résout les obstacles.
    
    \item \textbf{Équipe de développement :}
    \begin{itemize}
        \item Gharsallah Islem - Backend, IA
        \item Ben Jemaa Mohamed Malek - Frontend Web, Mobile
        \item Hammi Youssef - Base de données, Tests
    \end{itemize}
\end{itemize}

\subsection{Cérémonies Scrum}

Nous avons mis en place les cérémonies suivantes :

\begin{enumerate}
    \item \textbf{Sprint Planning :} Réunion de planification au début de chaque sprint pour définir les objectifs et sélectionner les user stories.
    
    \item \textbf{Daily Scrum :} Réunions quotidiennes de 15 minutes pour synchroniser l'équipe.
    
    \item \textbf{Sprint Review :} Démonstration des fonctionnalités développées à la fin du sprint.
    
    \item \textbf{Sprint Retrospective :} Analyse de ce qui a bien fonctionné et des axes d'amélioration.
\end{enumerate}

\subsection{Artefacts Scrum}

\begin{itemize}
    \item \textbf{Product Backlog :} Liste priorisée de toutes les fonctionnalités du projet.
    
    \item \textbf{Sprint Backlog :} Sous-ensemble du Product Backlog sélectionné pour le sprint en cours.
    
    \item \textbf{Increment :} Version potentiellement livrable du produit à la fin de chaque sprint.
\end{itemize}

\section{Organisation des Sprints}

Le projet est divisé en \textbf{6 sprints} de 2 semaines chacun :

\begin{table}[htbp]
\centering
\caption{Planification des sprints}
\label{tab:sprints}
\begin{tabular}{|c|l|l|}
\hline
\textbf{Sprint} & \textbf{Thème} & \textbf{Objectifs principaux} \\
\hline
1 & Fondation \& Authentification & Setup, Auth, Gestion utilisateurs \\
\hline
2 & Gestion des Produits & Catalogue, Recherche, Admin produits \\
\hline
3 & Panier \& Commandes & Panier, Checkout, Gestion commandes \\
\hline
4 & Paiement \& Inventaire & Stripe, Gestion stocks, Réapprovisionnement \\
\hline
5 & Module IA & Reconnaissance images, Recommandations \\
\hline
6 & Chat, Livraison \& Validation & Support, Suivi livraison, Tests finaux \\
\hline
\end{tabular}
\end{table}

\imagePlaceholder{gantt_chart}{Diagramme de Gantt du projet}

\section{Environnement de Développement}

\subsection{Outils de Développement}

\begin{table}[htbp]
\centering
\caption{Outils de développement utilisés}
\label{tab:outils}
\begin{tabular}{|l|l|l|}
\hline
\textbf{Catégorie} & \textbf{Outil} & \textbf{Version} \\
\hline
IDE Backend & IntelliJ IDEA & 2024.x \\
\hline
IDE Frontend & Visual Studio Code & 1.85+ \\
\hline
IDE Python & PyCharm / VS Code & 2024.x \\
\hline
SGBD & PostgreSQL & 15.x \\
\hline
Admin BDD & pgAdmin & 4.x \\
\hline
API Testing & Postman & 10.x \\
\hline
Versioning & Git & 2.x \\
\hline
Repository & GitHub & - \\
\hline
Conteneurs & Docker & 24.x \\
\hline
\end{tabular}
\end{table}

\subsection{Configuration des Environnements}

Notre projet utilise trois environnements distincts :

\begin{enumerate}
    \item \textbf{Développement :} Configuration locale pour le développement et les tests unitaires.
    
    \item \textbf{Staging :} Environnement de pré-production pour les tests d'intégration.
    
    \item \textbf{Production :} Environnement final déployé via Docker.
\end{enumerate}

\section{Architecture Globale du Système}

\subsection{Architecture en Couches}

Notre application suit une architecture en couches classique :

\imagePlaceholder{architecture_globale}{Architecture globale du système}

\begin{enumerate}
    \item \textbf{Couche Présentation :} 
    \begin{itemize}
        \item Application Web (Angular 18)
        \item Application Mobile (Kotlin Android native)
    \end{itemize}
    
    \item \textbf{Couche API :}
    \begin{itemize}
        \item Backend Spring Boot (REST API)
        \item Module IA FastAPI
    \end{itemize}
    
    \item \textbf{Couche Métier :}
    \begin{itemize}
        \item Services Spring Boot
        \item Logique de recommandation
    \end{itemize}
    
    \item \textbf{Couche Données :}
    \begin{itemize}
        \item PostgreSQL (base principale)
        \item Modèles TensorFlow (IA)
    \end{itemize}
\end{enumerate}

\subsection{Diagramme de Déploiement}

\imagePlaceholder{diagramme_deploiement}{Diagramme de déploiement UML}

\section{Stack Technologique Détaillée}

\subsection{Frontend}

\begin{table}[htbp]
\centering
\caption{Technologies Frontend}
\label{tab:frontend_tech}
\begin{tabular}{|l|l|}
\hline
\textbf{Technologie} & \textbf{Utilisation} \\
\hline
Angular 18 & Framework principal \\
\hline
TypeScript & Langage de programmation \\
\hline
Angular Material & Composants UI \\
\hline
RxJS & Programmation réactive \\
\hline
SCSS & Stylisation \\
\hline
Leaflet & Cartographie \\
\hline
Chart.js & Graphiques analytics \\
\hline
\end{tabular}
\end{table}

\subsection{Backend}

\begin{table}[htbp]
\centering
\caption{Technologies Backend}
\label{tab:backend_tech}
\begin{tabular}{|l|l|}
\hline
\textbf{Technologie} & \textbf{Utilisation} \\
\hline
Spring Boot 3 & Framework principal \\
\hline
Java 17 & Langage de programmation \\
\hline
Spring Security & Sécurité, JWT \\
\hline
Spring Data JPA & ORM, accès données \\
\hline
Hibernate & Implémentation JPA \\
\hline
WebSocket (STOMP) & Communication temps réel \\
\hline
Lombok & Réduction boilerplate \\
\hline
\end{tabular}
\end{table}

\subsection{Module IA}

\begin{table}[htbp]
\centering
\caption{Technologies Module IA}
\label{tab:ia_tech}
\begin{tabular}{|l|l|}
\hline
\textbf{Technologie} & \textbf{Utilisation} \\
\hline
Python 3.10 & Langage de programmation \\
\hline
FastAPI & Framework API \\
\hline
TensorFlow 2.12 & Deep learning \\
\hline
Keras & API haut niveau \\
\hline
NumPy & Calcul numérique \\
\hline
Pillow & Traitement images \\
\hline
\end{tabular}
\end{table}

\section{Gestion de la Qualité}

\subsection{Standards de Code}

Nous avons adopté les pratiques suivantes :

\begin{itemize}
    \item \textbf{Clean Code :} Nommage explicite, fonctions courtes, commentaires pertinents
    \item \textbf{Design Patterns :} Repository, Service, Factory, Singleton
    \item \textbf{SOLID Principles :} Respect des principes de conception orientée objet
\end{itemize}

\subsection{Revue de Code}

Chaque fonctionnalité fait l'objet d'une Pull Request sur GitHub, revue par au moins un autre membre de l'équipe avant fusion.

\subsection{Tests}

Notre stratégie de tests comprend :

\begin{itemize}
    \item Tests unitaires (JUnit, Jasmine)
    \item Tests d'intégration (Spring Boot Test)
    \item Tests API (Postman)
    \item Tests manuels de validation
\end{itemize}

\section{Conclusion}

L'adoption de la méthodologie Scrum et la mise en place d'un environnement de développement structuré nous ont permis de gérer efficacement ce projet complexe. Les six sprints planifiés couvrent l'ensemble des fonctionnalités de la plateforme, de l'authentification jusqu'au système de livraison.


% Sprint 1 - Fondation & Authentification
% ============================================================
% CHAPITRE 4 : SPRINT 1 - FONDATION & AUTHENTIFICATION
% ============================================================

\chapter{Sprint 1 : Fondation et Authentification}

\section{Presentation du Sprint}

\subsection{Objectifs}

Le premier sprint constitue la fondation de notre projet. Il vise a mettre en place l'infrastructure technique et le systeme d'authentification securise.

\begin{table}[htbp]
\centering
\caption{Fiche du Sprint 1}
\rowcolors{2}{lightgray}{white}
\begin{tabular}{|l|l|}
\hline
\rowcolor{darkblue}
\tableheader{Attribut} & \tableheader{Valeur} \\
\hline
Duree & 2 semaines \\
\hline
Theme & Fondation et Authentification \\
\hline
Priorite & Haute \\
\hline
\end{tabular}
\end{table}

\subsection{Objectifs Detailles}

\begin{enumerate}
    \item Configuration de l'environnement de developpement (Angular, Spring Boot, PostgreSQL)
    \item Mise en place de l'architecture du projet selon le pattern MVC
    \item Developpement du systeme d'authentification (inscription, connexion, deconnexion)
    \item Gestion des utilisateurs et des roles (CLIENT, ADMIN, DRIVER, SUPPORT, SUPER\_ADMIN)
    \item Securisation des API avec JWT (JSON Web Token)
    \item Verification d'email et reinitialisation de mot de passe
\end{enumerate}

\section{Sprint Backlog}

\begin{table}[htbp]
\centering
\caption{Sprint Backlog - Sprint 1}
\rowcolors{2}{lightgray}{white}
\begin{tabular}{|c|p{7cm}|c|c|}
\hline
\rowcolor{darkblue}
\tableheader{ID} & \tableheader{User Story} & \tableheader{Priorite} & \tableheader{Points} \\
\hline
US-1.1 & En tant qu'utilisateur, je veux creer un compte avec mon email et mot de passe & Haute & 5 \\
\hline
US-1.2 & En tant qu'utilisateur, je veux me connecter de maniere securisee & Haute & 5 \\
\hline
US-1.3 & En tant qu'utilisateur, je veux reinitialiser mon mot de passe oublie & Moyenne & 3 \\
\hline
US-1.4 & En tant qu'admin, je veux gerer les roles des utilisateurs & Haute & 5 \\
\hline
US-1.5 & En tant qu'utilisateur, je veux verifier mon email & Moyenne & 3 \\
\hline
US-1.6 & En tant qu'utilisateur, je veux mettre a jour mon profil & Moyenne & 3 \\
\hline
\end{tabular}
\end{table}

\section{Analyse et Conception}

\subsection{Diagramme de Cas d'Utilisation}

Le diagramme de cas d'utilisation suivant presente les fonctionnalites d'authentification et de gestion des utilisateurs. Il identifie les acteurs principaux : le Visiteur (non authentifie), l'Utilisateur authentifie, l'Administrateur avec privileges etendus, et le Systeme Email pour les notifications.

\begin{figure}[H]
\centering
\includegraphics[width=0.95\textwidth]{cas utilisation 1.png}
\caption{Diagramme de cas d'utilisation - Module Authentification}
\label{fig:sprint1_usecase}
\end{figure}

\textbf{Acteurs identifies :}
\begin{itemize}
    \item \textbf{\textcolor{darkblue}{Visiteur :}} Utilisateur non authentifie pouvant s'inscrire ou se connecter
    \item \textbf{\textcolor{darkblue}{Utilisateur :}} Utilisateur authentifie avec acces aux fonctionnalites de base
    \item \textbf{\textcolor{darkblue}{Administrateur :}} Utilisateur avec privileges de gestion complete
    \item \textbf{\textcolor{darkblue}{Systeme :}} Responsable de l'envoi d'emails et de la validation des tokens
\end{itemize}

\subsection{Diagramme de Classes}

Le diagramme de classes ci-dessous presente l'architecture des entites liees a la gestion des utilisateurs et a l'authentification.

\begin{figure}[H]
\centering
\includegraphics[width=0.95\textwidth]{classe 1.png}
\caption{Diagramme de classes - Entites User et Role}
\label{fig:sprint1_class}
\end{figure}

\textbf{Description des entites :}
\begin{itemize}
    \item \textbf{\textcolor{navyblue}{User :}} Entite principale representant un utilisateur du systeme. Elle contient les attributs : id (UUID), email, password (hash BCrypt), fullName, phone, address, profilePicture, isActive, isEmailVerified, emailVerificationToken, passwordResetToken, et les timestamps de creation/modification.
    \item \textbf{\textcolor{navyblue}{Role :}} Entite definissant les differents roles possibles dans le systeme. Les roles disponibles sont : CLIENT (client standard), ADMIN (administrateur), DRIVER (livreur), SUPPORT (agent de support), et SUPER\_ADMIN (super administrateur avec tous les privileges). Relation ManyToOne avec User.
\end{itemize}

\subsection{Diagramme de Sequence}

Le diagramme de sequence suivant illustre le processus complet de connexion avec generation et validation du token JWT.

\begin{figure}[H]
\centering
\includegraphics[width=0.95\textwidth]{sequence 1.png}
\caption{Diagramme de sequence - Processus de connexion JWT}
\label{fig:sprint1_sequence}
\end{figure}

\textbf{Description du flux de connexion :}
\begin{enumerate}
    \item L'utilisateur saisit ses identifiants (email, mot de passe) dans le formulaire
    \item Le frontend Angular envoie une requete POST vers /api/auth/login
    \item Le backend verifie les identifiants avec BCrypt
    \item Si valides, un token JWT est genere avec les claims (sub, role, exp)
    \item Le token est stocke dans localStorage cote client
    \item Les requetes suivantes incluent le token dans l'en-tete Authorization
\end{enumerate}

\section{Realisation}

\subsection{Architecture d'Authentification}

L'architecture d'authentification mise en place repose sur plusieurs composants :
\begin{itemize}
    \item \textbf{\textcolor{steelblue}{SecurityConfig :}} Configuration Spring Security avec routes publiques et protegees
    \item \textbf{\textcolor{steelblue}{JwtAuthenticationFilter :}} Filtre interceptant chaque requete pour valider le JWT
    \item \textbf{\textcolor{steelblue}{AuthService :}} Logique metier d'inscription, connexion et reinitialisation
\end{itemize}

\subsection{Gestion des Roles}

\begin{table}[htbp]
\centering
\caption{Roles et permissions}
\rowcolors{2}{lightgray}{white}
\begin{tabular}{|l|p{7cm}|}
\hline
\rowcolor{darkblue}
\tableheader{Role} & \tableheader{Permissions} \\
\hline
CLIENT & Acces au catalogue, panier, commandes, profil \\
\hline
ADMIN & Gestion produits, commandes, utilisateurs, analytics \\
\hline
DRIVER & Acces aux livraisons assignees, mise a jour des statuts \\
\hline
SUPPORT & Acces au chat support, gestion des reclamations \\
\hline
SUPER\_ADMIN & Tous les privileges, gestion des admins, configuration systeme \\
\hline
\end{tabular}
\end{table}

\section{Bilan du Sprint}

\begin{table}[htbp]
\centering
\caption{Bilan Sprint 1}
\rowcolors{2}{lightgray}{white}
\begin{tabular}{|l|c|}
\hline
\rowcolor{darkblue}
\tableheader{Metrique} & \tableheader{Valeur} \\
\hline
User Stories planifiees & 6 \\
\hline
User Stories terminees & 6 \\
\hline
Points planifies & 24 \\
\hline
Points realises & 24 \\
\hline
Velocite & 100\% \\
\hline
\end{tabular}
\end{table}

Le Sprint 1 a ete complete avec succes, etablissant une base solide pour les sprints suivants.


% Sprint 2 - Gestion des Produits
% ============================================================
% CHAPITRE 5 : SPRINT 2 - GESTION DES PRODUITS
% ============================================================

\chapter{Sprint 2 : Gestion des Produits et Catalogue}

\section{Presentation du Sprint}

\subsection{Objectifs}

Le deuxieme sprint se concentre sur le developpement du catalogue de produits, incluant les fonctionnalites de recherche, de filtrage et l'interface d'administration pour la gestion des produits.

\begin{table}[htbp]
\centering
\caption{Fiche du Sprint 2}
\rowcolors{2}{lightgray}{white}
\begin{tabular}{|l|l|}
\hline
\rowcolor{darkblue}
\tableheader{Attribut} & \tableheader{Valeur} \\
\hline
Duree & 2 semaines \\
\hline
Theme & Gestion des Produits \\
\hline
Priorite & Haute \\
\hline
\end{tabular}
\end{table}

\section{Sprint Backlog}

\begin{table}[htbp]
\centering
\caption{Sprint Backlog - Sprint 2}
\rowcolors{2}{lightgray}{white}
\begin{tabular}{|c|p{7cm}|c|c|}
\hline
\rowcolor{darkblue}
\tableheader{ID} & \tableheader{User Story} & \tableheader{Priorite} & \tableheader{Points} \\
\hline
US-2.1 & En tant qu'utilisateur, je veux parcourir les produits par categorie & Haute & 5 \\
\hline
US-2.2 & En tant qu'utilisateur, je veux rechercher des produits par mot-cle & Haute & 5 \\
\hline
US-2.3 & En tant qu'utilisateur, je veux filtrer les produits par prix et marque & Haute & 5 \\
\hline
US-2.4 & En tant qu'admin, je veux ajouter de nouveaux produits & Haute & 8 \\
\hline
US-2.5 & En tant qu'admin, je veux modifier et supprimer des produits & Haute & 5 \\
\hline
US-2.6 & En tant qu'admin, je veux gerer les categories et marques & Moyenne & 5 \\
\hline
US-2.7 & En tant qu'utilisateur, je veux voir les details d'un produit & Haute & 3 \\
\hline
\end{tabular}
\end{table}

\section{Analyse et Conception}

\subsection{Diagramme de Cas d'Utilisation}

Le diagramme suivant presente les fonctionnalites de gestion du catalogue differenciees par acteur.

\begin{figure}[H]
\centering
\includegraphics[width=0.95\textwidth]{cas utilisation 2.png}
\caption{Diagramme de cas d'utilisation - Gestion des Produits}
\label{fig:sprint2_usecase}
\end{figure}

\textbf{Cas d'utilisation principaux :}
\begin{itemize}
    \item \textbf{\textcolor{darkblue}{Parcourir le catalogue :}} Navigation dans les produits avec pagination
    \item \textbf{\textcolor{darkblue}{Rechercher des produits :}} Recherche textuelle par nom ou description
    \item \textbf{\textcolor{darkblue}{Filtrer par criteres :}} Application de filtres (categorie, marque, prix)
    \item \textbf{\textcolor{darkblue}{Gerer les produits :}} Operations CRUD sur les produits (Admin)
\end{itemize}

\subsection{Diagramme de Classes}

Ce diagramme presente la structure des entites produit avec leurs relations.

\begin{figure}[H]
\centering
\includegraphics[width=0.95\textwidth]{classe 2.png}
\caption{Diagramme de classes - Entites Product, Category, Brand et Vehicle}
\label{fig:sprint2_class}
\end{figure}

\textbf{Description des entites :}
\begin{itemize}
    \item \textbf{\textcolor{navyblue}{Product :}} Entite centrale representant une piece automobile. Attributs : id (UUID), name, description, price, stock, model, year, compatibility, imageUrl, isActive, timestamps. Relations ManyToOne avec Category et Brand, et ManyToMany avec Vehicle.
    \item \textbf{\textcolor{navyblue}{Category :}} Classification hierarchique des produits (Freinage, Moteur, Electricite, Suspension). Attributs : id, name, description.
    \item \textbf{\textcolor{navyblue}{Brand :}} Marques des pieces automobiles (Bosch, Valeo, Brembo). Attributs : id, name, country, description, logoUrl.
    \item \textbf{\textcolor{navyblue}{Vehicle :}} Vehicule compatible avec les pieces. Attributs : id, brand, model, year, engineType. Un produit peut etre compatible avec plusieurs vehicules.
    \item \textbf{\textcolor{navyblue}{ProductImage :}} Images additionnelles d'un produit. Attributs : id, imageUrl, isPrimary, sortOrder. Relation ManyToOne vers Product.
\end{itemize}

\subsection{Diagramme de Sequence}

Le diagramme illustre le flux de recherche avec filtres multiples.

\begin{figure}[H]
\centering
\includegraphics[width=0.95\textwidth]{sequence 2.png}
\caption{Diagramme de sequence - Recherche et filtrage de produits}
\label{fig:sprint2_sequence}
\end{figure}

\section{Realisation}

\subsection{Architecture du Catalogue}

\begin{itemize}
    \item \textbf{\textcolor{steelblue}{ProductController :}} Endpoints REST pour consultation et administration
    \item \textbf{\textcolor{steelblue}{ProductService :}} Logique metier incluant recherche avec Specifications JPA
    \item \textbf{\textcolor{steelblue}{ProductRepository :}} Interface Spring Data JPA avec methodes personnalisees
\end{itemize}

\section{Bilan du Sprint}

\begin{table}[htbp]
\centering
\caption{Bilan Sprint 2}
\rowcolors{2}{lightgray}{white}
\begin{tabular}{|l|c|}
\hline
\rowcolor{darkblue}
\tableheader{Metrique} & \tableheader{Valeur} \\
\hline
User Stories terminees & 7/7 \\
\hline
Points realises & 36 \\
\hline
Velocite & 100\% \\
\hline
\end{tabular}
\end{table}


% Sprint 3 - Panier & Commandes
% ============================================================
% CHAPITRE 6 : SPRINT 3 - PANIER & COMMANDES
% ============================================================

\chapter{Sprint 3 : Panier et Gestion des Commandes}

\section{Presentation du Sprint}

\subsection{Objectifs}

Le troisieme sprint se concentre sur le developpement des fonctionnalites de panier d'achat et de gestion des commandes.

\begin{table}[htbp]
\centering
\caption{Fiche du Sprint 3}
\rowcolors{2}{lightgray}{white}
\begin{tabular}{|l|l|}
\hline
\rowcolor{darkblue}
\tableheader{Attribut} & \tableheader{Valeur} \\
\hline
Duree & 2 semaines \\
\hline
Theme & Panier et Commandes \\
\hline
Priorite & Haute \\
\hline
\end{tabular}
\end{table}

\section{Sprint Backlog}

\begin{table}[htbp]
\centering
\caption{Sprint Backlog - Sprint 3}
\rowcolors{2}{lightgray}{white}
\begin{tabular}{|c|p{7cm}|c|c|}
\hline
\rowcolor{darkblue}
\tableheader{ID} & \tableheader{User Story} & \tableheader{Priorite} & \tableheader{Points} \\
\hline
US-3.1 & En tant qu'utilisateur, je veux ajouter des produits a mon panier & Haute & 5 \\
\hline
US-3.2 & En tant qu'utilisateur, je veux modifier les quantites dans mon panier & Haute & 3 \\
\hline
US-3.3 & En tant qu'utilisateur, je veux supprimer des articles de mon panier & Haute & 2 \\
\hline
US-3.4 & En tant qu'utilisateur, je veux passer une commande & Haute & 8 \\
\hline
US-3.5 & En tant qu'utilisateur, je veux voir l'historique de mes commandes & Haute & 5 \\
\hline
US-3.6 & En tant qu'admin, je veux gerer les statuts des commandes & Haute & 5 \\
\hline
US-3.7 & En tant qu'utilisateur, je veux recevoir une confirmation de commande & Moyenne & 3 \\
\hline
\end{tabular}
\end{table}

\section{Analyse et Conception}

\subsection{Diagramme de Cas d'Utilisation}

Le diagramme suivant presente l'ensemble des fonctionnalites liees au panier et aux commandes.

\begin{figure}[H]
\centering
\includegraphics[width=0.95\textwidth]{cas d'utlisation 3.png}
\caption{Diagramme de cas d'utilisation - Panier et Commandes}
\label{fig:sprint3_usecase}
\end{figure}

\textbf{Cas d'utilisation - Panier :}
\begin{itemize}
    \item \textbf{\textcolor{darkblue}{Ajouter au panier :}} Ajout d'un produit avec quantite specifiee
    \item \textbf{\textcolor{darkblue}{Modifier quantite :}} Augmentation ou diminution de la quantite
    \item \textbf{\textcolor{darkblue}{Supprimer du panier :}} Retrait d'un article du panier
    \item \textbf{\textcolor{darkblue}{Consulter panier :}} Visualisation du contenu et du total
    \item \textbf{\textcolor{darkblue}{Vider panier :}} Suppression de tous les articles
\end{itemize}

\textbf{Cas d'utilisation - Commande :}
\begin{itemize}
    \item \textbf{\textcolor{darkblue}{Passer commande :}} Finalisation de l'achat avec adresse de livraison
    \item \textbf{\textcolor{darkblue}{Consulter historique :}} Visualisation de toutes les commandes passees
    \item \textbf{\textcolor{darkblue}{Suivre commande :}} Consultation du statut actuel d'une commande
    \item \textbf{\textcolor{darkblue}{Annuler commande :}} Annulation d'une commande en attente
\end{itemize}

\textbf{Cas d'utilisation - Administration :}
\begin{itemize}
    \item \textbf{\textcolor{darkblue}{Lister commandes :}} Vue d'ensemble de toutes les commandes
    \item \textbf{\textcolor{darkblue}{Modifier statut :}} Changement du statut d'une commande
    \item \textbf{\textcolor{darkblue}{Confirmer commande :}} Validation d'une commande en attente
\end{itemize}

\subsection{Diagramme de Classes}

Le diagramme suivant illustre les relations entre les entites du module panier et commandes.

\begin{figure}[H]
\centering
\includegraphics[width=0.95\textwidth]{classe 3.png}
\caption{Diagramme de classes - Entites Cart, Order, OrderItem et Product}
\label{fig:sprint3_class}
\end{figure}

\textbf{Description des entites :}
\begin{itemize}
    \item \textbf{\textcolor{navyblue}{Cart :}} Panier d'achat d'un utilisateur. Attributs : id, createdAt, updatedAt. Relation OneToOne avec User, OneToMany avec CartItem. Methodes : getTotalPrice(), getTotalItems().
    \item \textbf{\textcolor{navyblue}{CartItem :}} Article dans le panier. Attributs : id, quantity, addedAt. Relations ManyToOne avec Cart et Product. Methode : getSubtotal().
    \item \textbf{\textcolor{navyblue}{Order :}} Commande validee. Attributs : id, totalPrice, status, deliveryAddress, trackingNumber, paymentMethod, paymentStatus, notes, timestamps. Relation ManyToOne avec User, OneToMany avec OrderItem.
    \item \textbf{\textcolor{navyblue}{OrderItem :}} Article d'une commande avec prix fige. Attributs : id, quantity, priceAtPurchase. Relations ManyToOne avec Order et Product.
    \item \textbf{\textcolor{navyblue}{Product :}} Produit reference par CartItem et OrderItem. Fournit les informations de prix et stock pour le calcul du panier.
\end{itemize}

\subsection{Diagramme de Sequence}

\begin{figure}[H]
\centering
\includegraphics[width=0.95\textwidth]{sequence 3.png}
\caption{Diagramme de sequence - Creation d'une commande}
\label{fig:sprint3_sequence}
\end{figure}

\section{Realisation}

\subsection{Cycle de Vie des Commandes}

\begin{table}[htbp]
\centering
\caption{Statuts des commandes}
\rowcolors{2}{lightgray}{white}
\begin{tabular}{|l|p{8cm}|}
\hline
\rowcolor{darkblue}
\tableheader{Statut} & \tableheader{Description} \\
\hline
PENDING & Commande creee, en attente de confirmation \\
\hline
CONFIRMED & Commande validee par l'administrateur \\
\hline
SHIPPED & Commande expediee, livraison en cours \\
\hline
DELIVERED & Commande livree au client \\
\hline
CANCELLED & Commande annulee \\
\hline
\end{tabular}
\end{table}

\section{Bilan du Sprint}

\begin{table}[htbp]
\centering
\caption{Bilan Sprint 3}
\rowcolors{2}{lightgray}{white}
\begin{tabular}{|l|c|}
\hline
\rowcolor{darkblue}
\tableheader{Metrique} & \tableheader{Valeur} \\
\hline
User Stories terminees & 7/7 \\
\hline
Points realises & 31 \\
\hline
Velocite & 100\% \\
\hline
\end{tabular}
\end{table}


% Sprint 4 - Paiement & Inventaire
% ============================================================
% CHAPITRE 7 : SPRINT 4 - PAIEMENT & INVENTAIRE
% ============================================================

\chapter{Sprint 4 : Paiement Stripe et Gestion d'Inventaire}

\section{Presentation du Sprint}

\subsection{Objectifs}

Le quatrieme sprint vise a integrer le systeme de paiement securise Stripe et a developper un module de gestion d'inventaire.

\begin{table}[htbp]
\centering
\caption{Fiche du Sprint 4}
\rowcolors{2}{lightgray}{white}
\begin{tabular}{|l|l|}
\hline
\rowcolor{darkblue}
\tableheader{Attribut} & \tableheader{Valeur} \\
\hline
Duree & 2 semaines \\
\hline
Theme & Paiement et Inventaire \\
\hline
Priorite & Haute \\
\hline
\end{tabular}
\end{table}

\section{Sprint Backlog}

\begin{table}[htbp]
\centering
\caption{Sprint Backlog - Sprint 4}
\rowcolors{2}{lightgray}{white}
\begin{tabular}{|c|p{7cm}|c|c|}
\hline
\rowcolor{darkblue}
\tableheader{ID} & \tableheader{User Story} & \tableheader{Priorite} & \tableheader{Points} \\
\hline
US-4.1 & En tant qu'utilisateur, je veux payer par carte bancaire & Haute & 8 \\
\hline
US-4.2 & En tant qu'utilisateur, je veux un paiement securise & Haute & 5 \\
\hline
US-4.3 & En tant qu'admin, je veux suivre les niveaux de stock & Haute & 5 \\
\hline
US-4.4 & En tant qu'admin, je veux des alertes de stock bas & Moyenne & 5 \\
\hline
US-4.5 & En tant qu'admin, je veux gerer les fournisseurs & Moyenne & 5 \\
\hline
US-4.6 & En tant qu'admin, je veux creer des commandes de reapprovisionnement & Moyenne & 5 \\
\hline
\end{tabular}
\end{table}

\section{Analyse et Conception}

\subsection{Diagramme de Cas d'Utilisation}

\begin{figure}[H]
\centering
\includegraphics[width=0.95\textwidth]{cas d'itlisation 4.png}
\caption{Diagramme de cas d'utilisation - Paiement et Inventaire}
\label{fig:sprint4_usecase}
\end{figure}

\textbf{Cas d'utilisation - Paiement :}
\begin{itemize}
    \item \textbf{\textcolor{darkblue}{Initier paiement :}} Demarrage du processus de paiement
    \item \textbf{\textcolor{darkblue}{Saisir carte :}} Entree des informations bancaires via Stripe
    \item \textbf{\textcolor{darkblue}{Confirmer paiement :}} Validation de la transaction
\end{itemize}

\textbf{Cas d'utilisation - Inventaire :}
\begin{itemize}
    \item \textbf{\textcolor{darkblue}{Consulter tableau de bord :}} Vue d'ensemble des stocks
    \item \textbf{\textcolor{darkblue}{Gerer mouvements :}} Enregistrement des entrees/sorties de stock
    \item \textbf{\textcolor{darkblue}{Gerer fournisseurs :}} CRUD sur les fournisseurs
    \item \textbf{\textcolor{darkblue}{Commander reapprovisionnement :}} Creation de commandes fournisseurs
\end{itemize}

\subsection{Diagramme de Classes}

\begin{figure}[H]
\centering
\includegraphics[width=0.95\textwidth]{classe 4.png}
\caption{Diagramme de classes - Payment, StockMovement, Supplier, PurchaseOrder et Product}
\label{fig:sprint4_class}
\end{figure}

\textbf{Description des entites :}
\begin{itemize}
    \item \textbf{\textcolor{navyblue}{Payment :}} Enregistrement d'un paiement. Attributs : id, amount, currency (TND), status (PENDING, COMPLETED, FAILED, REFUNDED), stripePaymentId, stripeSessionId, paymentMethod, timestamps. Relation OneToOne avec Order.
    \item \textbf{\textcolor{navyblue}{Order :}} Commande associee au paiement. Le paiement reference la commande pour laquelle il a ete effectue.
    \item \textbf{\textcolor{navyblue}{StockMovement :}} Historique des mouvements de stock. Attributs : id, quantity, type (IN, OUT, ADJUSTMENT), reason, reference, createdBy, createdAt. Relation ManyToOne avec Product.
    \item \textbf{\textcolor{navyblue}{Product :}} Produit dont le stock est affecte par les mouvements. Le stock est mis a jour automatiquement.
    \item \textbf{\textcolor{navyblue}{Supplier :}} Fournisseur de pieces. Attributs : id, name, email, phone, address, contactPerson, isActive.
    \item \textbf{\textcolor{navyblue}{PurchaseOrder :}} Commande de reapprovisionnement. Attributs : id, orderNumber, status, totalAmount, expectedDelivery, notes. Relation ManyToOne avec Supplier, OneToMany avec PurchaseOrderItem.
    \item \textbf{\textcolor{navyblue}{PurchaseOrderItem :}} Article d'une commande fournisseur. Attributs : id, quantity, unitPrice, receivedQuantity. Relation ManyToOne avec Product.
\end{itemize}

\subsection{Diagramme de Sequence}

\begin{figure}[H]
\centering
\includegraphics[width=0.95\textwidth]{sequence 4.png}
\caption{Diagramme de sequence - Paiement avec Stripe Checkout}
\label{fig:sprint4_sequence}
\end{figure}

\section{Realisation}

\subsection{Integration Stripe}

\begin{itemize}
    \item \textbf{\textcolor{steelblue}{Stripe Checkout :}} Interface hebergee par Stripe, conforme PCI-DSS
    \item \textbf{\textcolor{steelblue}{Webhooks :}} Endpoint securise pour confirmation asynchrone
    \item \textbf{\textcolor{steelblue}{Mode Test :}} Cles API de test pour simulation
\end{itemize}

\section{Bilan du Sprint}

\begin{table}[htbp]
\centering
\caption{Bilan Sprint 4}
\rowcolors{2}{lightgray}{white}
\begin{tabular}{|l|c|}
\hline
\rowcolor{darkblue}
\tableheader{Metrique} & \tableheader{Valeur} \\
\hline
User Stories terminees & 6/6 \\
\hline
Points realises & 33 \\
\hline
Velocite & 100\% \\
\hline
\end{tabular}
\end{table}


% Sprint 5 - Module IA
% ============================================================
% CHAPITRE 8 : SPRINT 5 - MODULE IA
% ============================================================

\chapter{Sprint 5 : Module d'Intelligence Artificielle}

\section{Presentation du Sprint}

\subsection{Objectifs}

Le cinquieme sprint est consacre au developpement du module d'Intelligence Artificielle, permettant la reconnaissance d'images et les recommandations personnalisees.

\begin{table}[htbp]
\centering
\caption{Fiche du Sprint 5}
\rowcolors{2}{lightgray}{white}
\begin{tabular}{|l|l|}
\hline
\rowcolor{darkblue}
\tableheader{Attribut} & \tableheader{Valeur} \\
\hline
Duree & 2 semaines \\
\hline
Theme & Module IA \\
\hline
Priorite & Haute \\
\hline
\end{tabular}
\end{table}

\section{Sprint Backlog}

\begin{table}[htbp]
\centering
\caption{Sprint Backlog - Sprint 5}
\rowcolors{2}{lightgray}{white}
\begin{tabular}{|c|p{7cm}|c|c|}
\hline
\rowcolor{darkblue}
\tableheader{ID} & \tableheader{User Story} & \tableheader{Priorite} & \tableheader{Points} \\
\hline
US-5.1 & En tant qu'utilisateur, je veux identifier une piece a partir d'une photo & Haute & 13 \\
\hline
US-5.2 & En tant qu'utilisateur, je veux des recommandations basees sur mon historique & Haute & 8 \\
\hline
US-5.3 & En tant qu'utilisateur, je veux decrire des symptomes et obtenir des suggestions & Moyenne & 8 \\
\hline
US-5.4 & En tant qu'admin, je veux voir les statistiques d'utilisation de l'IA & Moyenne & 5 \\
\hline
\end{tabular}
\end{table}

\section{Analyse et Conception}

\subsection{Diagramme de Cas d'Utilisation}

\begin{figure}[H]
\centering
\includegraphics[width=0.95\textwidth]{cas d'utilisation 5.png}
\caption{Diagramme de cas d'utilisation - Module IA}
\label{fig:sprint5_usecase}
\end{figure}

\textbf{Cas d'utilisation principaux :}
\begin{itemize}
    \item \textbf{\textcolor{darkblue}{Identifier piece :}} L'utilisateur uploade une photo pour identification
    \item \textbf{\textcolor{darkblue}{Obtenir recommandations :}} Suggestions basees sur l'historique d'achat
    \item \textbf{\textcolor{darkblue}{Analyser symptomes :}} Description textuelle pour suggestion de pieces
\end{itemize}

\subsection{Diagramme de Classes - Backend}

\begin{figure}[H]
\centering
\includegraphics[width=0.95\textwidth]{classe 5.png}
\caption{Diagramme de classes - Recommendation, UserActivity et entites associees}
\label{fig:sprint5_class}
\end{figure}

\textbf{Description des entites :}
\begin{itemize}
    \item \textbf{\textcolor{navyblue}{Recommendation :}} Enregistrement d'une recommandation IA. Attributs : id, confidenceScore, symptoms, aiResponse, isViewed, createdAt. Relations ManyToOne avec User et Product.
    \item \textbf{\textcolor{navyblue}{RecommendationType :}} Enumeration definissant les types de recommandations : AI\_PART\_RECOGNITION, SYMPTOM\_ANALYSIS, SIMILAR\_PRODUCTS, PURCHASE\_HISTORY, TRENDING, FREQUENTLY\_BOUGHT.
    \item \textbf{\textcolor{navyblue}{UserActivity :}} Historique des activites utilisateur. Attributs : id, searchQuery, timestamp. Relations ManyToOne avec User et Product.
    \item \textbf{\textcolor{navyblue}{ActivityType :}} Enumeration des types d'activites : VIEW, SEARCH, ADD\_TO\_CART, PURCHASE, WISHLIST, AI\_IDENTIFY.
    \item \textbf{\textcolor{navyblue}{User :}} Utilisateur pour lequel les recommandations sont generees et les activites enregistrees.
    \item \textbf{\textcolor{navyblue}{Product :}} Produit recommande ou concerne par l'activite utilisateur.
\end{itemize}

\subsection{Diagramme de Classes - Module Python}

\begin{figure}[H]
\centering
\includegraphics[width=0.95\textwidth]{classe ai 5.png}
\caption{Diagramme de classes - Module IA Python (FastAPI, EfficientNetB0)}
\label{fig:sprint5_class_ai}
\end{figure}

\textbf{Composants IA :}
\begin{itemize}
    \item \textbf{\textcolor{steelblue}{FastAPIApp :}} Application Python exposant les endpoints /predict et /health
    \item \textbf{\textcolor{steelblue}{EfficientNetB0Model :}} Modele CNN pre-entraine avec 50 classes de pieces
    \item \textbf{\textcolor{steelblue}{ImagePreprocessor :}} Traitement des images (resize 224x224, normalisation)
\end{itemize}

\subsection{Diagramme de Sequence}

\begin{figure}[H]
\centering
\includegraphics[width=0.95\textwidth]{sequence 5.png}
\caption{Diagramme de sequence - Reconnaissance d'image IA}
\label{fig:sprint5_sequence}
\end{figure}

\section{Realisation}

\subsection{Modele de Classification}

\begin{table}[htbp]
\centering
\caption{Specifications du modele IA}
\rowcolors{2}{lightgray}{white}
\begin{tabular}{|l|l|}
\hline
\rowcolor{darkblue}
\tableheader{Caracteristique} & \tableheader{Valeur} \\
\hline
Architecture & EfficientNetB0 (pre-entraine ImageNet) \\
\hline
Nombre de classes & 50 types de pieces \\
\hline
Taille d'entree & 224 x 224 pixels \\
\hline
Accuracy (validation) & 94.2\% \\
\hline
Temps d'inference & ~50ms \\
\hline
\end{tabular}
\end{table}

\section{Bilan du Sprint}

\begin{table}[htbp]
\centering
\caption{Bilan Sprint 5}
\rowcolors{2}{lightgray}{white}
\begin{tabular}{|l|c|}
\hline
\rowcolor{darkblue}
\tableheader{Metrique} & \tableheader{Valeur} \\
\hline
User Stories terminees & 4/4 \\
\hline
Points realises & 34 \\
\hline
Velocite & 100\% \\
\hline
\end{tabular}
\end{table}


% Sprint 6 - Chat, Livraison & Validation
% ============================================================
% CHAPITRE 9 : SPRINT 6 - CHAT, LIVRAISON & VALIDATION
% ============================================================

\chapter{Sprint 6 : Chat Support, Livraison et Validation Finale}

\section{Presentation du Sprint}

\subsection{Objectifs}

Le sixieme et dernier sprint couvre le developpement du systeme de chat en temps reel, du suivi de livraison avec cartographie, et la validation finale du projet.

\begin{table}[htbp]
\centering
\caption{Fiche du Sprint 6}
\rowcolors{2}{lightgray}{white}
\begin{tabular}{|l|l|}
\hline
\rowcolor{darkblue}
\tableheader{Attribut} & \tableheader{Valeur} \\
\hline
Duree & 2 semaines \\
\hline
Theme & Chat, Livraison, Validation \\
\hline
Priorite & Haute \\
\hline
\end{tabular}
\end{table}

\section{Sprint Backlog}

\begin{table}[htbp]
\centering
\caption{Sprint Backlog - Sprint 6}
\rowcolors{2}{lightgray}{white}
\begin{tabular}{|c|p{6.5cm}|c|c|}
\hline
\rowcolor{darkblue}
\tableheader{ID} & \tableheader{User Story} & \tableheader{Priorite} & \tableheader{Points} \\
\hline
US-6.1 & En tant qu'utilisateur, je veux contacter le support par chat & Haute & 8 \\
\hline
US-6.2 & En tant qu'utilisateur, je veux suivre ma livraison sur une carte & Haute & 8 \\
\hline
US-6.3 & En tant qu'admin, je veux gerer les livraisons & Haute & 5 \\
\hline
US-6.4 & En tant que livreur, je veux mettre a jour le statut de livraison & Moyenne & 5 \\
\hline
US-6.5 & En tant qu'utilisateur, je veux soumettre une reclamation & Moyenne & 5 \\
\hline
US-6.6 & Tests d'integration et validation finale & Haute & 8 \\
\hline
\end{tabular}
\end{table}

\section{Analyse et Conception}

\subsection{Diagramme de Cas d'Utilisation}

\begin{figure}[H]
\centering
\includegraphics[width=0.95\textwidth]{cas d'utlisation 6.png}
\caption{Diagramme de cas d'utilisation - Chat et Livraison}
\label{fig:sprint6_usecase}
\end{figure}

\textbf{Acteurs :}
\begin{itemize}
    \item \textbf{\textcolor{darkblue}{Utilisateur :}} Chat avec support, suivi livraison, reclamations
    \item \textbf{\textcolor{darkblue}{Support :}} Repondre aux conversations, traiter reclamations
    \item \textbf{\textcolor{darkblue}{Administrateur :}} Assigner livreurs, gerer livraisons
    \item \textbf{\textcolor{darkblue}{Livreur :}} Mettre a jour statut, transmettre position
\end{itemize}

\subsection{Diagramme de Classes}

\begin{figure}[H]
\centering
\includegraphics[width=0.95\textwidth]{classe 6.png}
\caption{Diagramme de classes - Conversation, Message, Delivery, Driver et entites associees}
\label{fig:sprint6_class}
\end{figure}

\textbf{Description des entites :}
\begin{itemize}
    \item \textbf{\textcolor{navyblue}{Conversation :}} Discussion entre un utilisateur et le support. Attributs : id, title, isActive, createdAt, updatedAt. Relations ManyToOne avec User, OneToMany avec Message.
    \item \textbf{\textcolor{navyblue}{Message :}} Message dans une conversation. Attributs : id, content, senderType (USER, SUPPORT, SYSTEM), isRead, createdAt. Relation ManyToOne avec Conversation et User.
    \item \textbf{\textcolor{navyblue}{User :}} Utilisateur initiant la conversation ou envoyant des messages.
    \item \textbf{\textcolor{navyblue}{Delivery :}} Livraison associee a une commande. Attributs : id, trackingNumber, status, address, driverName, driverPhone, currentLatitude, currentLongitude, estimatedDelivery, pickedUpAt, deliveredAt. Relation OneToOne avec Order, ManyToOne avec Driver.
    \item \textbf{\textcolor{navyblue}{Order :}} Commande pour laquelle la livraison est creee. Chaque commande expediee genere une livraison.
    \item \textbf{\textcolor{navyblue}{Driver :}} Livreur assigne a la livraison. Attributs : id, name, phone, email, vehicleType, licensePlate, isAvailable, currentLatitude, currentLongitude.
    \item \textbf{\textcolor{navyblue}{DriverLocation :}} Historique des positions du livreur. Attributs : id, latitude, longitude, timestamp, speed, heading. Relation ManyToOne avec Driver.
    \item \textbf{\textcolor{navyblue}{Reclamation :}} Plainte soumise par un utilisateur. Attributs : id, subject, description, status, response, attachmentUrl, createdAt, resolvedAt. Relations ManyToOne avec User et Order.
\end{itemize}

\subsection{Diagramme de Sequence}

\begin{figure}[H]
\centering
\includegraphics[width=0.95\textwidth]{séquence 6.png}
\caption{Diagramme de sequence - Suivi de livraison en temps reel}
\label{fig:sprint6_sequence}
\end{figure}

\section{Realisation}

\subsection{Chat en Temps Reel}

\begin{itemize}
    \item \textbf{\textcolor{steelblue}{WebSocket avec STOMP :}} Communication bidirectionnelle
    \item \textbf{\textcolor{steelblue}{SockJS :}} Fallback pour navigateurs incompatibles
    \item Messages persistants avec historique
\end{itemize}

\subsection{Suivi de Livraison}

\begin{table}[htbp]
\centering
\caption{Statuts de livraison}
\rowcolors{2}{lightgray}{white}
\begin{tabular}{|l|p{7cm}|}
\hline
\rowcolor{darkblue}
\tableheader{Statut} & \tableheader{Description} \\
\hline
PROCESSING & Livraison creee, en attente d'assignation \\
\hline
IN\_TRANSIT & Livreur en route vers la destination \\
\hline
OUT\_FOR\_DELIVERY & Livreur proche de l'adresse \\
\hline
DELIVERED & Colis livre avec succes \\
\hline
FAILED & Echec de livraison \\
\hline
\end{tabular}
\end{table}

\section{Validation Finale}

\begin{table}[htbp]
\centering
\caption{Scenarios de validation}
\rowcolors{2}{lightgray}{white}
\begin{tabular}{|p{8cm}|c|}
\hline
\rowcolor{darkblue}
\tableheader{Scenario} & \tableheader{Resultat} \\
\hline
Parcours complet : inscription, achat, paiement, livraison & \checkmark \\
\hline
Chat temps reel entre utilisateur et support & \checkmark \\
\hline
Suivi de livraison avec mise a jour en temps reel & \checkmark \\
\hline
Identification IA et ajout au panier & \checkmark \\
\hline
\end{tabular}
\end{table}

\section{Bilan du Sprint}

\begin{table}[htbp]
\centering
\caption{Bilan Sprint 6}
\rowcolors{2}{lightgray}{white}
\begin{tabular}{|l|c|}
\hline
\rowcolor{darkblue}
\tableheader{Metrique} & \tableheader{Valeur} \\
\hline
User Stories terminees & 6/6 \\
\hline
Points realises & 39 \\
\hline
Velocite & 100\% \\
\hline
\end{tabular}
\end{table}

Ce dernier sprint a permis de completer toutes les fonctionnalites et de valider l'integration de l'ensemble des modules.


% Tests & Validation
% ============================================================
% CHAPITRE 10 : TESTS ET VALIDATION
% ============================================================

\chapter{Tests et Validation}

\section{Introduction}

Ce chapitre presente la strategie de tests mise en place pour assurer la qualite et la fiabilite de notre plateforme. Nous avons adopte une approche multi-niveaux couvrant les tests unitaires, d'integration et de validation utilisateur.

\section{Strategie de Tests}

\subsection{Pyramide des Tests}

Notre strategie suit la pyramide des tests classique :

\diagrammePlaceholder{test_pyramid}{Pyramide des tests}

\begin{enumerate}
    \item \textbf{Tests Unitaires (Base) :} Testent les composants individuels en isolation
    \item \textbf{Tests d'Integration (Milieu) :} Verifient l'interaction entre les modules
    \item \textbf{Tests E2E (Sommet) :} Valident les parcours utilisateur complets
\end{enumerate}

\section{Tests Unitaires}

\subsection{Backend - JUnit et Mockito}

Les tests unitaires du backend couvrent :

\begin{itemize}
    \item \textbf{Services :} Logique metier (AuthService, ProductService, OrderService, etc.)
    \item \textbf{Utilitaires :} Fonctions de generation de tokens, validation, etc.
    \item \textbf{Entites :} Methodes de calcul et validations
\end{itemize}

\textbf{Outils utilises :}
\begin{itemize}
    \item JUnit 5 pour les assertions et le cycle de vie des tests
    \item Mockito pour le mocking des dependances
    \item Spring Boot Test pour le contexte d'application
\end{itemize}

\subsection{Frontend - Jasmine et Karma}

Les tests unitaires Angular couvrent :

\begin{itemize}
    \item \textbf{Services :} Appels HTTP, gestion d'etat
    \item \textbf{Composants :} Logique de presentation, interactions
    \item \textbf{Pipes et Directives :} Transformations de donnees
\end{itemize}

\subsection{Module IA - Pytest}

Les tests du module Python couvrent :

\begin{itemize}
    \item Verification du chargement du modele
    \item Tests de prediction avec images de reference
    \item Validation des endpoints API (health, predict)
\end{itemize}

\section{Tests d'Integration}

\subsection{Tests API}

Les tests d'integration verifient le bon fonctionnement des endpoints REST :

\begin{table}[htbp]
\centering
\caption{Resume des tests API}
\begin{tabular}{|l|c|c|}
\hline
\textbf{Module} & \textbf{Nombre de tests} & \textbf{Reussis} \\
\hline
Authentication & 8 & 8 \\
\hline
Products & 12 & 12 \\
\hline
Cart & 6 & 6 \\
\hline
Orders & 10 & 10 \\
\hline
Payments & 4 & 4 \\
\hline
Delivery & 8 & 8 \\
\hline
Chat & 6 & 6 \\
\hline
AI & 4 & 4 \\
\hline
\textbf{Total} & \textbf{58} & \textbf{58} \\
\hline
\end{tabular}
\end{table}

\subsection{Tests avec Postman}

Une collection Postman complete a ete creee pour tester manuellement et automatiquement toutes les API. La collection comprend :

\begin{itemize}
    \item Variables d'environnement (URLs, tokens)
    \item Pre-request scripts pour l'authentification automatique
    \item Tests automatises pour valider les reponses
    \item Collection Runner pour l'execution en lot
\end{itemize}

\section{Tests d'Acceptation Utilisateur}

\subsection{Scenarios de Test}

Les scenarios suivants ont ete valides avec des utilisateurs reels :

\begin{table}[htbp]
\centering
\caption{Scenarios de test utilisateur}
\begin{tabular}{|p{9cm}|c|}
\hline
\textbf{Scenario} & \textbf{Resultat} \\
\hline
Un utilisateur s'inscrit, verifie son email et se connecte & \checkmark \\
\hline
Un utilisateur recherche et filtre des produits par categorie & \checkmark \\
\hline
Un utilisateur ajoute des produits au panier et passe commande & \checkmark \\
\hline
Un utilisateur effectue un paiement via Stripe & \checkmark \\
\hline
Un utilisateur utilise l'IA pour identifier une piece & \checkmark \\
\hline
Un utilisateur suit sa livraison en temps reel & \checkmark \\
\hline
Un utilisateur contacte le support via chat & \checkmark \\
\hline
Un admin gere les produits et commandes & \checkmark \\
\hline
\end{tabular}
\end{table}

\section{Couverture de Code}

\begin{table}[htbp]
\centering
\caption{Couverture de code par module}
\begin{tabular}{|l|c|c|c|}
\hline
\textbf{Module} & \textbf{Lignes} & \textbf{Branches} & \textbf{Fonctions} \\
\hline
Backend Services & 78\% & 72\% & 85\% \\
\hline
Backend Controllers & 82\% & 75\% & 90\% \\
\hline
Frontend Services & 75\% & 68\% & 80\% \\
\hline
Module IA & 70\% & 65\% & 75\% \\
\hline
\end{tabular}
\end{table}

\section{Validation des Fonctionnalites}

\begin{table}[htbp]
\centering
\caption{Validation par plateforme}
\begin{tabular}{|l|c|c|c|}
\hline
\textbf{Module} & \textbf{Web} & \textbf{Mobile} & \textbf{Admin} \\
\hline
Authentification & \checkmark & \checkmark & \checkmark \\
\hline
Catalogue produits & \checkmark & \checkmark & \checkmark \\
\hline
Panier et commandes & \checkmark & \checkmark & \checkmark \\
\hline
Paiement Stripe & \checkmark & \checkmark & - \\
\hline
Module IA & \checkmark & \checkmark & - \\
\hline
Chat support & \checkmark & \checkmark & \checkmark \\
\hline
Suivi livraison & \checkmark & \checkmark & \checkmark \\
\hline
\end{tabular}
\end{table}

\section{Conclusion}

La strategie de tests mise en place nous a permis d'assurer un niveau de qualite eleve pour notre plateforme. Avec un taux de reussite de 100\% sur les tests automatises et une validation complete des scenarios utilisateur, nous avons confiance dans la stabilite et la fiabilite de l'application.


% Conclusion
% ============================================================
% CHAPITRE 11 : CONCLUSION ET PERSPECTIVES
% ============================================================

\chapter{Conclusion et Perspectives}

\section{Synthèse des Réalisations}

Ce projet de fin d'études a abouti à la conception et au développement d'une plateforme e-commerce intelligente pour les pièces détachées automobiles, intégrant des technologies d'Intelligence Artificielle innovantes.

\subsection{Objectifs Atteints}

Nous avons réussi à atteindre l'ensemble des objectifs fixés au début du projet :

\begin{table}[htbp]
\centering
\caption{Bilan des objectifs}
\label{tab:objectifs_atteints}
\begin{tabular}{|p{8cm}|c|}
\hline
\textbf{Objectif} & \textbf{Statut} \\
\hline
Système d'authentification sécurisé (JWT, RBAC) & \checkmark \\
\hline
Catalogue de produits avec recherche et filtres & \checkmark \\
\hline
Gestion du panier et des commandes & \checkmark \\
\hline
Intégration du paiement Stripe & \checkmark \\
\hline
Module IA de reconnaissance d'images (94\% précision) & \checkmark \\
\hline
Système de recommandation personnalisé & \checkmark \\
\hline
Chat de support en temps réel & \checkmark \\
\hline
Suivi de livraison avec cartographie & \checkmark \\
\hline
Application mobile utilisateur & \checkmark \\
\hline
\end{tabular}
\end{table}

\subsection{Métriques du Projet}

\begin{table}[htbp]
\centering
\caption{Statistiques du projet}
\label{tab:stats_projet}
\begin{tabular}{|l|c|}
\hline
\textbf{Métrique} & \textbf{Valeur} \\
\hline
Nombre de sprints & 6 \\
\hline
Durée totale & 12 semaines \\
\hline
User Stories livrées & 40 \\
\hline
Points Story réalisés & 197 \\
\hline
Entités backend & 30 \\
\hline
Endpoints API & 85+ \\
\hline
Composants frontend & 50+ \\
\hline
Tests automatisés & 58 \\
\hline
Couverture de code moyenne & 76\% \\
\hline
\end{tabular}
\end{table}

\subsection{Fonctionnalités Clés Développées}

\subsubsection{Module d'Intelligence Artificielle}

Le module IA représente l'innovation principale de notre plateforme :

\begin{itemize}
    \item \textbf{Reconnaissance d'images} : Classification de 50 types de pièces auto avec une précision de 94.2\%
    \item \textbf{Analyse de symptômes} : Suggestions de pièces basées sur la description de problèmes
    \item \textbf{Recommandations personnalisées} : Algorithme hybride combinant filtrage collaboratif et content-based
\end{itemize}

\subsubsection{Système de Livraison Intelligent}

\begin{itemize}
    \item Simulation de mouvement du livreur en temps réel
    \item Intégration OpenRouteService pour les itinéraires
    \item Visualisation sur carte Leaflet
    \item Notifications WebSocket
\end{itemize}

\subsubsection{Architecture Robuste}

\begin{itemize}
    \item Architecture microservices avec séparation Backend/IA
    \item API RESTful sécurisée
    \item Base de données PostgreSQL optimisée
    \item Frontend Angular 18 avec Signals
\end{itemize}

\section{Défis Techniques Surmontés}

\subsection{Défis Majeurs}

\begin{table}[htbp]
\centering
\caption{Défis techniques rencontrés}
\label{tab:defis}
\begin{tabular}{|p{4cm}|p{7cm}|}
\hline
\textbf{Défi} & \textbf{Solution Adoptée} \\
\hline
Précision du modèle IA & Utilisation de EfficientNetB0 pré-entraîné avec fine-tuning \\
\hline
Communication temps réel & WebSocket avec STOMP pour chat et suivi \\
\hline
Géocodage d'adresses en Tunisie & Base de coordonnées locales + fallback ORS \\
\hline
Sécurisation des paiements & Mode test Stripe + webhooks sécurisés \\
\hline
Performances avec gros volumes & Pagination, lazy loading, indexation BDD \\
\hline
\end{tabular}
\end{table}

\section{Limitations Actuelles}

Malgré les fonctionnalités implémentées, certaines limitations persistent :

\begin{itemize}
    \item \textbf{Paiements simulés} : Le système utilise le mode test de Stripe, les vrais paiements nécessiteraient une validation supplémentaire.
    
    \item \textbf{Livraison simulée} : Le mouvement du livreur est simulé, l'intégration avec de vrais livreurs GPS reste à faire.
    
    \item \textbf{Dataset IA limité} : Le modèle est entraîné sur 50 classes, une extension serait nécessaire pour couvrir toutes les pièces.
\end{itemize}

\section{Perspectives d'Amélioration}

\subsection{Court Terme (3-6 mois)}

\begin{enumerate}
    \item \textbf{Passage en production}
    \begin{itemize}
        \item Activation du mode live Stripe
        \item Déploiement sur serveur dédié avec HTTPS
        \item Configuration des environnements de production
    \end{itemize}
    
    \item \textbf{Amélioration du modèle IA}
    \begin{itemize}
        \item Extension du dataset à 200+ classes
        \item Entraînement avec augmentation de données
        \item Optimisation pour mobile (TensorFlow Lite)
    \end{itemize}
    
    \item \textbf{Intégration livreurs réels}
    \begin{itemize}
        \item Application mobile pour livreurs
        \item Tracking GPS temps réel
        \item Optimisation des itinéraires
    \end{itemize}
\end{enumerate}

\subsection{Moyen Terme (6-12 mois)}

\begin{enumerate}
    \item \textbf{Amélioration application mobile}
    \begin{itemize}
        \item Optimisation des performances Kotlin
        \item Notifications push Firebase
        \item Mode hors ligne pour le catalogue
    \end{itemize}
    
    \item \textbf{Chatbot IA}
    \begin{itemize}
        \item Assistant conversationnel pour le diagnostic
        \item Intégration NLP avancé
        \item Support multilingue (Français, Arabe, Anglais)
    \end{itemize}
    
    \item \textbf{Marketplace multi-vendeurs}
    \begin{itemize}
        \item Intégration de vendeurs tiers
        \item Système de commissions
        \item Tableau de bord vendeur
    \end{itemize}
\end{enumerate}

\subsection{Long Terme (12+ mois)}

\begin{enumerate}
    \item \textbf{Expansion régionale}
    \begin{itemize}
        \item Déploiement multi-pays (Maghreb)
        \item Adaptation aux réglementations locales
        \item Réseau de partenaires logistiques
    \end{itemize}
    
    \item \textbf{IA prédictive}
    \begin{itemize}
        \item Prédiction des besoins de maintenance
        \item Alertes personnalisées basées sur le véhicule
        \item Analyse de la durée de vie des pièces
    \end{itemize}
    
    \item \textbf{Réalité augmentée}
    \begin{itemize}
        \item Identification de pièces via caméra AR
        \item Guide d'installation visuel
        \item Vérification de compatibilité 3D
    \end{itemize}
\end{enumerate}

\section{Compétences Acquises}

Ce projet nous a permis de développer et renforcer de nombreuses compétences :

\subsection{Compétences Techniques}

\begin{itemize}
    \item Développement full-stack (Angular, Spring Boot)
    \item Intelligence Artificielle et Machine Learning
    \item Architecture microservices
    \item Bases de données relationnelles
    \item API RESTful et WebSocket
    \item Intégrations tierces (Stripe, ORS)
\end{itemize}

\subsection{Compétences Méthodologiques}

\begin{itemize}
    \item Méthodologie Scrum
    \item Gestion de projet agile
    \item Tests et assurance qualité
    \item Documentation technique
\end{itemize}

\subsection{Compétences Transversales}

\begin{itemize}
    \item Travail en équipe
    \item Résolution de problèmes complexes
    \item Communication technique
    \item Gestion du temps et des priorités
\end{itemize}

\section{Conclusion Générale}

Ce projet de fin d'études nous a permis de mettre en pratique les connaissances acquises durant notre formation et d'explorer de nouvelles technologies innovantes. La plateforme développée répond aux besoins identifiés et propose une solution originale grâce à l'intégration de l'Intelligence Artificielle.

Le développement en suivant la méthodologie Scrum nous a permis de livrer un produit fonctionnel et de qualité, capable d'évoluer pour répondre aux besoins futurs du marché des pièces automobiles.

Cette expérience constitue une base solide pour notre carrière professionnelle et nous a préparés aux défis du développement logiciel moderne.


% Annexes
\appendix
% ============================================================
% ANNEXES
% ============================================================

\chapter{Annexe A : Schema de Base de Donnees}

\section{Diagramme Entite-Relation}

\diagrammePlaceholder{erd_complete}{Diagramme entite-relation complet de la base de donnees}

\section{Description des Tables Principales}

\subsection{Table Users}

\begin{table}[htbp]
\centering
\caption{Structure de la table users}
\begin{tabular}{|l|l|l|}
\hline
\textbf{Colonne} & \textbf{Type} & \textbf{Description} \\
\hline
id & UUID & Identifiant unique \\
\hline
email & VARCHAR(255) & Adresse email (unique) \\
\hline
password & VARCHAR(255) & Mot de passe hashe \\
\hline
full\_name & VARCHAR(255) & Nom complet \\
\hline
phone & VARCHAR(20) & Numero de telephone \\
\hline
role\_id & UUID & Reference vers le role \\
\hline
is\_active & BOOLEAN & Compte actif \\
\hline
\end{tabular}
\end{table}

\subsection{Table Products}

\begin{table}[htbp]
\centering
\caption{Structure de la table products}
\begin{tabular}{|l|l|l|}
\hline
\textbf{Colonne} & \textbf{Type} & \textbf{Description} \\
\hline
id & UUID & Identifiant unique \\
\hline
name & VARCHAR(255) & Nom du produit \\
\hline
description & TEXT & Description detaillee \\
\hline
price & DECIMAL(10,2) & Prix unitaire \\
\hline
stock & INTEGER & Quantite en stock \\
\hline
category\_id & UUID & Reference vers la categorie \\
\hline
brand\_id & UUID & Reference vers la marque \\
\hline
\end{tabular}
\end{table}

\subsection{Table Orders}

\begin{table}[htbp]
\centering
\caption{Structure de la table orders}
\begin{tabular}{|l|l|l|}
\hline
\textbf{Colonne} & \textbf{Type} & \textbf{Description} \\
\hline
id & UUID & Identifiant unique \\
\hline
user\_id & UUID & Reference vers l'utilisateur \\
\hline
total\_price & DECIMAL(10,2) & Prix total de la commande \\
\hline
status & VARCHAR(50) & Statut de la commande \\
\hline
delivery\_address & TEXT & Adresse de livraison \\
\hline
tracking\_number & VARCHAR(50) & Numero de suivi \\
\hline
\end{tabular}
\end{table}

% ============================================================

\chapter{Annexe B : Guide d'Installation}

\section{Prerequis}

Pour installer et executer le projet, les outils suivants sont necessaires :

\begin{itemize}
    \item Java 17 ou superieur
    \item Node.js 18 ou superieur
    \item Python 3.10 ou superieur
    \item PostgreSQL 15 ou superieur
    \item Maven 3.8 ou superieur
    \item Git
\end{itemize}

\section{Installation du Backend}

\begin{enumerate}
    \item Cloner le repository du projet
    \item Configurer les variables d'environnement dans application.properties
    \item Creer la base de donnees PostgreSQL
    \item Executer mvn clean install pour compiler
    \item Lancer avec mvn spring-boot:run
\end{enumerate}

\section{Installation du Frontend}

\begin{enumerate}
    \item Naviguer vers le dossier frontend-web
    \item Executer npm install pour installer les dependances
    \item Lancer avec ng serve pour le developpement
    \item Compiler avec ng build --configuration production pour la production
\end{enumerate}

\section{Installation du Module IA}

\begin{enumerate}
    \item Naviguer vers le dossier ai-module
    \item Creer un environnement virtuel Python
    \item Installer les dependances avec pip install -r requirements.txt
    \item Lancer avec uvicorn src.api.main:app --port 5000
\end{enumerate}


\end{document}
